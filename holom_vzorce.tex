\chapter{Holomorfni funkce}

\section*{Vzorce}
\subsection*{Derivace}
\begin{equation}
\label{eq:derivace}
f'(z) = \lim_{h \to 0} \frac{f(z+h)-f(z)}{h}
\end{equation}

\subsection*{Cauchy-Riemannovy podminky}
Jsou nutne pro existenci derivace.

\begin{equation}
\label{eq:cauchy}
\begin{array}{ccc}
\frac{\delta u}{\delta x} (x,y) & = & \frac{\delta v}{\delta y} (x,y)\\
\frac{\delta u}{\delta y} (x,y) & = & -\frac{\delta v}{\delta x} (x,y)\\
\end{array}
\end{equation}
$u$ je realna slozka funkce $f(z)$\\
$v$ je imaginarni slozka funkce $f(z)$\\

Parcialni (po castech) derivace (podle $x$):
$$f'(z) = \frac{\delta u}{\delta x} (x,y) + j\frac{\delta v}{\delta x} (x,y)$$

Komplexni funkce $f(z)$ ma v bode $z = x+jy$ derivaci prave tehdy, kdyz jeji slozky $u$ a $v$ splnuji Cauchy-Riemannovy podminky a maji obe \underline{totalni diferencial} v bode (x,y) (spojitost parcialnich derivaci + CR podminky $\Rightarrow$ existence derivace)

\subsection*{Definice pojmu}
\begin{itemize}
\item Funkce f je \underline{holomorfni v otevrene mnozine} $G \subset \mathbb{C}$, jestlize ma derivaci v kazdem bode mnoziny $G$
\item Funkce f je \underline{holomorfni v bode} $z_0$, je-li holomorfni v nejakem okoli $z_0$
\item Holomorfni fce v otevrene mnozine $G$ se nazyva \underline{konformni}, jestlize $f'(z) \neq 0$ pro vsechna $z \in G$
\end{itemize}

\subsection*{Laplaceova rovnice}
\begin{equation}
\label{eq:laplace_r}
\frac{\delta ^2 u}{\delta x^2} + \frac{\delta ^2 u}{\delta y^2} = 0
\end{equation}
(obdobne pro slozku $v$)

Funkce splnujici laplaceovu rovnici se nazyvaji \underline{harmonicke}.

Realna a imaginarni slozka holomorfni funkce je harmonicka.

\subsection*{Elementarni funkce}
\subsubsection*{Afinni funkce}
\begin{equation}
\label{eq:afinni}
f(z) = az+b
\end{equation}

\subsubsection*{Linearni lomenne zobrazeni (M\" obiova transformace)}
Jedna se o slozeni afinnich funkci - slouzi ke kruhove inverzi vuci jednotkovemu kruhu a osove soumernosti dle realne osy.

Linearni lomenne zobrazeni zachovava zobecnene kruznice a body inverzni vuci nim.

\begin{equation}
\label{eq:mobiova}
f(z) = \frac{az+b}{cz+d}
\end{equation}
Pozor! $ad-bc \neq 0$ a $c \neq 0$, musime overovat!
Specialni pripady:

$$f(\infty) = \frac{a}{c}\; \; \; f\left( -\frac{d}{c}\right) = \infty$$

\subsubsection*{Polynomy}
$$f(z) = a_nz^n + a_{n-1}z^{n-1}+ \dots a_0\; \; \; a_n \neq 0$$
Zakladni veta algebry: Kazdy polynom stupne alespon jedna ma alespon jeden komplexni koren.

\subsubsection*{Racionalni funkce}
$$f(z) = \frac{p(z)}{q(z)}$$
je holomorfni, ba konformni v $\mathbb{C}$ az na koreny polynomu $q$

\subsubsection*{Exponencialni funkce}
$$e^z = e^{x+jy} = e^x\cdot e^jy = e^x \cdot (\operatorname{cos}(y)+j\operatorname{sin}(y)) = e^{Re}\cdot (\operatorname{cos}(Im)+j\operatorname{sin}(Im))$$

Komplexni exponencialni funkce neni prosta, tzn \textbf{nema inverzi!}.

\underline{Eulerova identita}: 

$e^{j\varphi} = \operatorname{cos}(\varphi)+j\operatorname{sin}(\varphi) \rightarrow e^{j\pi} = -1$

Dulezite:  (plyne z vyse popsaneho)
$$e^{j\pi} +1 = 0$$

\subsubsection*{Logaritmus}
Plati pro $z \neq 0$

$$\operatorname{Ln}z = \{ ln|z|+j\operatorname{arg}z + 2k\pi j | k\in \mathbb{Z}\}$$
Dulezite je si povsimnout, ze imaginarni cast funkce je periodicka! Rozlisujeme tedy:

\textbf{Hlavni vetev logaritmu} pro $k = 0$:
$$\operatorname{ln} z = ln|z|+j\operatorname{arg}z$$

\textit{Poznamka: povsimnete si rozdilu pouziti velkeho a maleho "L"/"l" v nazvech logaritmu!}

\subsubsection*{Goniometricke a hyperbolicke funkce}

$$\operatorname{cos}z = \frac{e^{jz}+e^{-jz}}{2}$$
$$\operatorname{sin}z = \frac{e^{jz}-e^{-jz}}{2j}$$
$$\operatorname{cosh}z = \frac{e^{z}+e^{-z}}{2}$$
$$\operatorname{sinh}z = \frac{e^{z}-e^{-z}}{2}$$

\textbf{Osbournova pravidla:}
Otoceni o 90$^\circ$:
$$\operatorname{cos}(jz) = \operatorname{cosh}(z)$$
$$\operatorname{sin}(jz) = j\operatorname{sinh}(z)$$

\textbf{Uzitecne vzorce pro upravy:}
$$\operatorname{cos}(x+jy) = \operatorname{cos}(x)\cdot \operatorname{cos}(jy)-\operatorname{sin}(x)\cdot \operatorname{sin}(jy)$$
$$\operatorname{sin}(x+jy) = \operatorname{sin}(x)\cdot \operatorname{cos}(jy)+\operatorname{cos}(x)\cdot \operatorname{sin}(jy)$$

\subsubsection*{Cyklometricke, mnohoznacne funkce}

V komplexnich funkci neni jednoduche urcit napr. arcsin. Pokud mame zadano, ze mame najit:
$\operatorname{Arcsin} z$ tak vysledek polozime roven $a$, tedy:
$$\operatorname{Arcsin} z =a $$
a prevedeme to tedy na reseni rovnice:
$$\operatorname{sin} a = z$$
A pak muzeme pouzit rozklad funkce sin na exponencialni fce a rovnici vyresit. Nutno podotknout, ze $a \in \mathbb{C}$

\subsection*{Krivkovy integral}

$$\int_C f(z) = f(\varphi(t))\cdot \varphi ' (t)$$
Kde $C$ a $\varphi(t)$ oznacuji krivku.

\textbf{Cauchyho veta:}
Necht $f$ je holomorfni na hladke Jordanove krivce $C$ a na jejım vnitrku. Potom je

$$\int_C f(z) dz = 0$$

Cauchyho vzorec:
\begin{equation}
\label{eq:cauchyho}
\int_C \frac{f(z)}{(z-z_0)} = 2\pi j f(z_0)
\end{equation}
