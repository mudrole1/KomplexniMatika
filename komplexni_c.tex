\chapter{Komplexni cisla}

\section*{Vzorce}

\subsubsection*{Tri tvary komplexniho cisla:}
\begin{enumerate}
  \item kartezsky $z=x+jy$
  \item goniometricky $z=|z|(\operatorname{cos}(\varphi)+j\operatorname{sin}(\varphi))$, kde 
  \begin{itemize}
    \item $z|$ je absolutni hodnota, oznacuje velikost a spocte se: $|z|=\sqrt{x^2+y^2}$, nebo take $|z|=\sqrt{z\cdot \bar{z}}$
    \item $\operatorname{cos}(\varphi)=\frac{Re\{z\}}{|z|} = \frac{x}{|z|}$
    \item $\operatorname{sin}(\varphi) = \frac{Im\{z\}}{|z|} = \frac{y}{|z|}$
    \item $\operatorname{tan}(\varphi) = \frac{\operatorname{sin}(\varphi)}{\operatorname{cos}(\varphi)}=\frac{Im\{z\}}{Re\{z\}}$
    \item $\varphi$ tez oznacovan jako argument $arg_z$: $\varphi = \operatorname{arctan}\left( \frac{Im\{z\}}{Re\{z\}}\right)$
    \item \textit{pozn.: tyto rovnice vychazi z jednoduche trojuhelnikove zavislosti, kde x,y jsou odvesny a $|z|$ je prepona. Uhel $\varphi$ je sviran mezi odvesnou x a preponou.}
  \end{itemize}
  \item exponencialni $z=|z|e^{j\varphi}$, kde $|z|$, $\varphi$ se spocte stejne jako v predchozim bode
\end{enumerate}

\subsubsection*{\label{sec:go_vz}Uzitecne vzorce pro praci s  goniometrickymi funkcemi}

Souctove vzorce
\begin{itemize}
\item $\operatorname{sin}(x+y)=\operatorname{sin}(x)\operatorname{cos}(y)+\operatorname{cos}(x)\operatorname{sin}(y)$
\item $\operatorname{sin}(x-y)=\operatorname{sin}(x)\operatorname{cos}(y)-\operatorname{cos}(x)\operatorname{sin}(y)$
\item $\operatorname{cos}(x+y)=\operatorname{cos}(x)\operatorname{cos}(y)-\operatorname{sin}(x)\operatorname{sin}(y)$
\item $\operatorname{cos}(x-y)=\operatorname{cos}(x)\operatorname{cos}(y)+\operatorname{sin}(x)\operatorname{sin}(y)$
\end{itemize}
\begin{equation}
\label{eq:sine}
sin(z) = \frac{e^{jz}-e^{-jz}}{2j}
\end{equation}
\begin{equation}
\label{eq:cose}
%TODO: doplnit
\end{equation}
\subsubsection*{Komplexne sdruzene cislo}
$\bar{z}$ oznacuje konjugaci, nebo-li cislo komplexne sdruzene. Pokud $z=x+jy$, pak $\bar{z}=x-jy$. Plati take:
$$\overline{zw}=\bar{z}\bar{w}$$
$$Re\{z\} = \frac{z+\bar{z}}{2}$$
$$Im\{z\} = \frac{z-\bar{z}}{2j}$$

\subsubsection*{Matematicke operace nad komplexnimi cisly}
\begin{itemize}
\item scitani je jako posun v rovine
\item nasobeni je otoceni o uhel $\varphi$ kolem pocatku, a pak stejnolehlost se stredem v pocatku a koeficientem $|a|$
\item deleni je otoceni o uhel $-\varphi$ a stejnolehlost s $\frac{1}{|a|}$
\item \textbf{porovnavani (tedy $z>w$) neni u komplexnich cisel definovano!}
\end{itemize}

\subsubsection*{Metrika bodu}
Z trojuhelnikove zavislost realne, imaginarni a absolutni hodnoty a zpusobu scitani dvou komplexnich cisel plyne nasledujici metrika bodu 
\begin{itemize}
\item $|Re\{z\}|,|Im\{z\}| \leq |z|$
\item $|z+w| \leq |z| + |w|$
\item $|z+w| \geq ||z|-|w||$
\item $|z| \leq |Re\{z\}|+|Im\{z\}|$
\end{itemize}

\subsubsection{Komplexni jednotka}
Tak se oznacuje komplexni cislo s $|z|=1$ a libovolnym uhlem, tedy takove komplexni cislo lezi kdekoliv na jednotkove kruznici.

\subsubsection*{Moivrova veta a binomicka rovnice}
Moivrova veta rika, ze:
\begin{equation}
\label{eq:moivrova}
z^n = |z|^n (\operatorname{cos}(n\varphi)+j\operatorname{sin}(n\varphi))
\end{equation}
Binomicka rovnice se pouze na nalezeni takoveho komplexniho cisla $a$, pro ktere plati:
\begin{equation}
\label{eq:binomicka}
z^n = a
\end{equation}
\textbf{pozor:}k vypoctenemu uhlu musime vzdy pridat $2\pi k$, abychom zajistili periodicnost

\subsubsection*{Geometrie}
Rovnice kruznice:
\begin{equation}
\label{eq:kruznice}
Az\cdot \bar{z}+\bar{a}z+a\bar{z}+C = 0; \; A \neq 0,\; C \in R; \; a\bar{a}-AC >0
\end{equation}
Z teto rovnice take muzeme ziskat kanonicky tvar:
\begin{equation}
\label{eq:kanonicky}
|z+\frac{a}{A}|=\sqrt{\frac{a\bar{a}-AC}{A^2}}
\end{equation}
Rovnice primky, tez osa dvou bodu
\begin{equation}
\label{eq:primka}
\bar{a}z+a\bar{z}+C=0; \; C\in R
\end{equation}
Apolloniovy kruznice: $\alpha,\; \beta$ jsou body lezici na jedne primce. Jeden lezi uvnitr kruznice, druhy vne.
\begin{equation}
\label{eq:apollo}
\frac{|z-\alpha |}{|z-\beta |}=\lambda
\end{equation}
Pokud dosadime za $z=x+jy$ ziskame tvar:
\begin{equation}
\label{eq:apollo_xy}
\left( x- \frac{\alpha_1 - \lambda^2 \beta_1}{1-\lambda^2}\right)+ \left( y - \frac{\alpha_2 - \lambda^2 \beta_2}{1-\lambda^2}\right)= r^2 
\end{equation}
Inverzni bod vuci primce: pouze osove sdruzeny\\
Inverzni bod vuci kruznici:
\begin{equation}
\label{eq:inverzni_kr}
|\alpha-a|\cdot |\beta - a| = r^2
\end{equation}
Kruhova inverze: je to zobrazeni, ktere k danemu bodu najde bod inverzni; a je stred kruznice
\begin{equation}
\label{eq:kruh_inverze}
f(z)=a+\frac{r^2}{\overline{z-a}}
\end{equation}
Riemannove sfera:
\begin{equation}
\label{eq:riemann}
x^2+y^2+\left( u- \frac{1}{2}\right)^2 = \frac{1}{4}
\end{equation}
Nebo take v jinem zapise:
\begin{equation}
\label{eq:rieamann_2}
x^2+y^2+u^2 = u
\end{equation}
Rovnice sfery:
\begin{equation}
\label{eq:sfera}
\Phi (x+jy) = \left[ \frac{x}{1+x^2+y^2}; \frac{y}{1+x^2+y^2};\frac{x^2+y^2}{1+x^2+y^2} \right]
\end{equation}
Stereograficka projekce: primku $N+t(z-N)=[t_x, t_y,1-t]$ dosadit do rovnice sfery.

\newpage

\section{Naleznete kartezsky tvar cisel}
\subsection{$z=\frac{1-j}{1+j}$}
Zlomek usmernim, tzn vynasobim ho takovym clenem, abych se zbavila komplexniho cisla ve jmenovateli. K tomu se perfektne hodi komplexne sdruzene cislo k jmenovateli, tedy nasobim clenem $\frac{\bar{z}}{\bar{z}}$
$$z=\frac{1-j}{1+j}\cdot \frac{1-j}{1-j} =\frac{(1-j)^2}{1+1}=\frac{1-2j-1}{2}=\frac{-2j}{2}=-j$$

\subsection{$z=j^n$}
Kdyz na to pujdeme hrubou silou, tak mame:
\begin{itemize}
\item $n=1$: $j$
\item $n=2$: $-1$
\item $n=3$: $-j$
\item $n=4$: $1$
\item $n=5 = n\cdot n^4$: $j\cdot1 = j$
\item $n=6$: opakuje se dale vzdy s peridou T=4
\end{itemize}

Tedy to muzeme nejak chytre zapsat. Vyuziji k tomu dalsi promenou $k=1,2,\dots $, abych vyjadrila periodu
\begin{itemize}
\item $j^{4k}=j$
\item $j^{4k+1}=-1$
\item $j^{4k+2}=-j$
\item $j^{4k+3}=1$
\end{itemize}

\subsection{$z=\frac{(2+j)(3+j)}{1-j}$}

Toto se resi opet usmernenim zlomku, opet vyuzijeme komplexne sdruzeneho cisla k jmenovateli:
$$z=\frac{(2+j)(3+j)}{1-j}\cdot \frac{1+j}{1+j}=\frac{(6-1+5j)(1-j)}{1+1}=\frac{5+10j-5}{2}=5j$$

\section{Jaka je absolutni hodnota a argument cisel:}
\begin{itemize}
\item $z=-1-j$; absolutni hodnota $|z|=\sqrt{(x^2+y^2)}= \sqrt{1+1}=\sqrt{2}$ nebo tez $|z|=\sqrt{z\cdot \bar{z}} = \sqrt{(-1-j)(-1+j)}=\sqrt{1+1}=\sqrt{2}$; argument (tim se mysli uhel $\varphi$): $arg_z = \varphi = \operatorname{arctan}\frac{Im\{z\}}{Re\{z\}}= \operatorname{arctan}\frac{-1}{-1} = \operatorname{arctan}(1) = \frac{\pi}{4}$; je dobre mit napsanou take tabulku pro uhly 30, 45, 60, 90 stupnu pro sin, cos, tan, cotg
\item $z=2+6j$; absolutni hodnota $|z|=\sqrt{2^2+6^2}=\sqrt{4+36}=\sqrt{40}=\sqrt{4\cdot 10}=2\sqrt{10}$; zde jsem to castecne odmocnila - rozlozila jsem cislo 40 na nasobeni takovych clenu, kde jeden z nich dokazi odmocnit na cele cislo. Argument $arg_z = \operatorname{arctan}\frac{6}{2}=\operatorname{arctan}(3)$; toto nevede na zadny cely uhel, muzeme to nechat vyjadrene takto.
\item $z=-2+6j$, absolutni hodnota je stejna jako v predchozim prikladu. Komplexni cislo lezi v II. kvadrantu (zaporna osa x, kladna osa y), takovy uhel se promita na zapornou cast funkce tan, tedy argument je $arg_z = - \operatorname{arctan}(3)$
\item ostatni obdobne
\end{itemize}

\section{Pomoci matematicke indukce dokazte Moivreovu formuli $(\operatorname{cos}(\varphi)+j\operatorname{sin}(\varphi))^n=\operatorname{cos}(n\varphi)+j\operatorname{sin}(n\varphi)$}

Matematicka indukce znamena, ze musime vzorec dokazat pro $n=1$ nejprve, a dale pro $n+1$, kde ale predpokladam, ze to plati pro $n$. Kdyz mi vyjde co ocekavam, muj predpoklad pro $n$ byl platny a tudiz mam hotovo.

 $n=1$ je reseni primocare, protoze rovnice ze zadani se proste rovnaji. Pro pripad $n+1$ vyuzijeme vlastnosti, ze $(a+b)^{n+1}=(a+b)^n \cdot (a+b)$, tedy v nasem pripade $$(\operatorname{cos}(\varphi)+j\operatorname{sin}(\varphi))^{n+1} = (\operatorname{cos}(\varphi)+j\operatorname{sin}(\varphi))^{n}\cdot (\operatorname{cos}(\varphi)+j\operatorname{sin}(\varphi)) = $$
 Jak rikam, predpokladame, ze pro $^n$ vzorec ze zadani plati, tedy ho pouzijeme:
 $$= (\operatorname{cos}(n\varphi)+j\operatorname{sin}(n\varphi)) \cdot (\operatorname{cos}(\varphi)+j\operatorname{sin}(\varphi)) = $$
 A ted zavorky roznasobime:
 $$ = \operatorname{cos}(n\varphi)\cdot \operatorname{cos}(\varphi)+j\operatorname{sin}(n\varphi)\cdot \operatorname{cos}(\varphi) + j\operatorname{sin}(\varphi)\cdot \operatorname{cos}(n\varphi) - \operatorname{sin}(n\varphi)\cdot \operatorname{sin}(\varphi) $$
 Ted aplikujeme ruzne souctove goniometricke vzorce, viz vypsane vzorce v uvodu, a to konkretne tyto dva:
  $$\operatorname{cos}(n\varphi)\cdot \operatorname{cos}(\varphi) - \operatorname{sin}(n\varphi)\cdot \operatorname{sin}(\varphi) = \operatorname{cos}(n\varphi + \varphi)$$
  $$j\operatorname{sin}(n\varphi)\cdot \operatorname{cos}(\varphi) + j\operatorname{sin}(\varphi)\cdot \operatorname{cos}(n\varphi) = j \operatorname{sin}(n\varphi+\varphi)$$
  
  Pokud je aplikujeme, dostavame (na leve strane mame porad to, co dokazujeme):
  $$(\operatorname{cos}(\varphi)+j\operatorname{sin}(\varphi))^{n+1} =\operatorname{cos}(n\varphi + \varphi)+j \operatorname{sin}(n\varphi+\varphi) = \operatorname{cos}((n+1)\varphi)+j\operatorname{sin}((n+1)\varphi) $$
  Coz presne odpovida tomu, co jsme ocekavali. Mame dokazano :)
 
 \section{Vypoctete vsechny hodnoty nasledujicich odmocnin}
 a) $\sqrt{1-j}$
 Cislo pod odmocninou oznacme napr. $z=1-j$. Proto cislo urcime modul (velikost) a argument (uhel). Tedy:
 $$|z|=\sqrt{1+1} = \sqrt{2}$$
$$\varphi=\operatorname{arg}(z) = \operatorname{arctg}\frac{-1}{1} = -\frac{\pi}{4}$$ 
  Dale celou odmocninu $\sqrt{z}$ dame rovnu nejake dalsi promene, treba $a$, tu chceme urcit.
 $$\sqrt{1-j}= a$$
 Odmocniny ale resit neumime, prevedeme to teda umocnenim na:
 $$(1-j) = a^2$$
 To prevedeme na goniometricky tvar komplexniho cisla:
 $$|z|\cdot (\operatorname{cos}(\varphi)+j \operatorname{sin}(\varphi)) = |a|^2 \cdot  (\operatorname{cos}(2\psi)+j \operatorname{sin}(2\psi))$$
 A prostym porovnanim clenu na leve a prave strane ziskame:
 $$|z|=|a|^2|$$
 $$a=\sqrt{|z|} = \sqrt{\sqrt{2}} = \sqrt[4]{2}$$
 U uhlu musime mit jeste napameti periodicnost a pridat $2k\pi$ k pokryti vsech existujicich uhlu a ne jen tech od $<0, 360>$, tedy:
 $$\varphi+2k\pi = 2\psi$$
 $$\psi = \frac{\varphi+2k\pi}{2}$$
 Takze napriklad pro $k=0$:
 $$\psi_0 = \frac{-\frac{\pi}{4}}{2} = -\frac{/pi}{8}$$
 Pro $k=1$
 $$\psi_1 = \frac{-\frac{\pi}{4}+2\pi}{2}=\frac{\frac{-\pi+8\pi}{4}}{2}=\frac{7\pi}{8}$$
 A tak dale... 
 
 b)-d) obdobne
 
 \section{Najdete vsechna reseni rovnice}
 a) $z^4=-1$
 Postup obdobny jako v predchozim prikladu. V prvnim kroku prevedeme na goniometricky tvar a porovname cleny:
 $$|z|^4\cdot (\operatorname{cos}(4\varphi)+j \operatorname{sin}(4\varphi)) = |-1|\cdot (\operatorname{cos}(\pi)+j \operatorname{sin}(\pi))$$
 $$|z|^4 = 1$$
 $$z = \sqrt[4]{1} = 1$$
 $$4\varphi = \pi+2k\pi$$
 $$\varphi = \frac{\pi (1+2k)}{4}$$
 Tedy:
 $$\psi_0 = \frac{\pi}{4}, \psi_1 = \frac{3\pi}{4}, \dots$$
 
Poznamka: uhel $2k\pi$ pridavame na opacnou stranu rovnice, nez co chceme vypocitat. Tady jsme chteli vypocitat $\varphi$, proto jsme to pridali napravo.

b) $z^7-z = 0$, zde nejprve musime upravit a vytknout jedno $z$:
$$z(z^6-1)= 0$$
coz se nam rozpadne na dve reseni:
$z = 0$ a $z^6 -1 = 0$. Prvni je vyreseno, druhe prevedeme na:
$$z^6 = 1$$
a resime stejne jako v predchozim bodu. Nesmime vsak u reseni zapomenout i na to extra $z=0$!

\section{Overte, ze plati:}
\subsection*{a) $|z_1\cdot z_2| = |z_1|\cdot |z_2|$}
Nejprve vypoctu levou stranu:
$$|z_1 \cdot z_2| = |(x_1+j y_1)\cdot (x_2+j y_2)| = |x_1 x_2 +j x_1 y_2 +j x_2 y_1 - y_1 y_2|=$$
V dalsim kroku k sobe dam realnou a imiginarni cast (abych v tom videla komplexni cislo).
$$=|(x_1 x_2 - y_1 y_2)+j(x_1 y_2 +x_2 y_1)|=$$
Velikost komplexniho cisla je $\sqrt{re^2 + im^2}$, tento vzorec aplikuje i zde:
$$=\sqrt{(x_1 x_2 - y_1 y_2)^2+(x_1 y_2+ x_2 y_1)^2}=$$
kdyz to roznasobim dle vzorce $(a-b)^2 = a^2 -2ab +b^2$ ziskam:
$$=\sqrt{(x_1^2 x_2^2 -2 x_1 x_2 y_1 y_2 + y_1^2 y_2^2)+(x_1^2 y_2^2 +2 x_1 x_2 y_1 y_2 +x_2^2 y_1^2)}=$$
Cleny $-2 x_1 x_2 y_1 y_2$ a $-2 x_1 x_2 y_1 y_2$ se odectou, zbyde tedy jen:

$$=\sqrt{x_1^2 x_2^2 +y_1^2 y_2^2 + x_1^2 y_2^2 + x_2^2 y_1^2}$$
To je uprava pro levou stranu. Pro pravou stranu plati:
$$|z_1|\cdot |z_2| = \sqrt{x_1^2+y_1^2}\cdot \sqrt{x_2^2 +y_2 ^2} = $$
Protoze je mezi odmocninami nasobeni, muzu to dat pod jednu odmocninu dle vzorce $\sqrt{a}\cdot \sqrt{b}=\sqrt{a\cdot b}$, tedy:
$$=\sqrt{(x_1^2+y_1^2)\cdot (x_2^2 +y_2 ^2)} = $$
Roznasobim a ziskam:
$$=\sqrt{x_1^2 x_2^2 +y_1^2 y_2^2 + x_1^2 y_2^2 + x_2^2 y_1^2}$$
Coz je vysledek prave strany a presne se shoduje s levou. Dokazali jsme tedy, ze rovnost plati.

\subsection*{b) $\lvert \frac{z_1}{z_2} \rvert = \frac{|z_1|}{|z_2|}$}
Opet nejprve pro levou stranu, postupujeme stejne, komplexni cislo rozepiseme na realnou a imiginarni cast a upravujeme do zblbnuti :)
$$\lvert \frac{x_1 + j y_1}{x_2 +j y_2} \rvert = $$
zde se hodi usmernit, tedy zbavit se komplexniho cisla ve jmenovateli. To udelam vynasobenim zlomkem, ktery ma komplexne sdruzene cislo jak v citateli, tak jmenovateli, neboli:
$$ = \lvert \frac{x_1 + j y_1}{x_2 +j y_2} \cdot \frac{x_2 -j y_2}{x_2 -j y_2} \rvert = \lvert \frac{(x_1 + j y_1)(x_2 -j y_2)}{x_2^2 - j^2 y_2^2} \rvert =$$
A pokracujeme v roznasobeni a upravach, rovnou preskupim cleny, aby realne a imaginarni casti byly u sebe:
$$= \lvert \frac{(x_1 x_2+y_1 y_2)+j(x_1 y_2 - x_2 y_1)}{x_2^2 + y_2^2} \rvert= \lvert \frac{(x_1 x_2+y_1 y_2)}{x_2^2 + y_2^2}+j \frac{x_1 y_2 - x_2 y_1}{x_2^2 + y_2^2} \rvert = $$
Pro nazornost jsem zlomek roztrhla, aby byla videt realne a imaginarni cast dobre. Dale vypocteme velikost:
$$=\sqrt{\frac{(x_1 x_2+y_1 y_2)^2}{(x_2^2 + y_2^2)^2}+\frac{(x_1 y_2 - x_2 y_1)^2}{(x_2^2 + y_2^2)^2}} = $$
Roznasobim a ziskam:
$$= \sqrt{\frac{x_1^2 x_2^2 +2 x_1 x_2 y_1 y_2 + y_1^2 y_2^2}{(x_2^2 + y_2^2)^2}+\frac{x_1^2 y_2^2 -2 x_1 x_2 y_1 y_2 +x_2^2 y_1^2}{(x_2^2 + y_2^2)^2}}=$$
Prevedu na spolecny jmenovatel, to je zde jednoduche, protoze jmenovatele jsou stejne. Nektere cleny v citateli se odectou.
$$=\sqrt{\frac{x_1^2 x_2^2 +y_1^2 y_2^2 + x_1^2 y_2^2 + x_2^2 y_1^2}{(x_2^2 + y_2^2)^2}}=$$
Coz z prvniho a ctvrteho clenu muzu vytknout $x_2^2$, z druheho a tretiho $y_2^2$, tedy:
$$=\sqrt{\frac{x_2^2(x_1^2 +y_1^2)+y_2^2(x_1^2+y_1^2)}{(x_2^2 + y_2^2)^2}}=$$
Ted lze vytknout jeste $(x_1^2+y_1^2)$, tedy:
$$=\sqrt{\frac{(x_1^2+y_1^2)(x_2^2+y_2^2)}{(x_2^2 + y_2^2)^2}}=$$
A muzeme jeste zkratit druhou zavorku s jmenovatele, ziskame tedy:
$$=\sqrt{\frac{(x_1^2+y_1^2)}{(x_2^2 + y_2^2)}}$$
Tim upravy pro levou stranu konci.

Upravy pro pravou stranu jsou jednodusi:
$$\frac{|z_1|}{|z_2|}=\frac{|x_1+j y_1|}{|x_2+j y_2|} = \frac{\sqrt{x_1^2+y_1^2}}{\sqrt{x_2^2+y_2^2}} = $$
Protoze je mezi odmocnina deleni, muzeme stejne jako u nasobeni prevest pod jednu odmocninu (neplati pro scitani a odecitani!)
$$=\sqrt{\frac{(x_1^2+y_1^2)}{(x_2^2+y_2^2)}}$$
coz je presne to stejne, co nam vyslo pro levou stranu. Mame hotovo a dokazano.

\subsection*{c)$\overline{z_1+z_2} = \overline{z_1}+\overline{z_2}$}
Opet upravime nejdrive levou stranu:
$$\overline{(x_1+j y_1)+(x_2+j y_2)} =$$
Jen preskupim cleny, realne k sobe, imaginarni k sobe, abych lepe videla nove komplexni cislo:
$$=\overline{(x_1+x_2)+j(y_1+y_2)}=$$
Cara nad vyrazem znamena konjugaci, komplexne sdruzene cislo. Tedy otoceni znameka pred imaginarni slozkou, tedy:
$$=(x_1+x_2) - j(y_1+y2)$$
Pro pravou stranu:
$$\overline{x_1+jy_1}+\bar{x_2+jy_2} = x_1 - jy_1 + x_2 - jy_2 = (x_1+x_2)-j(y_1+y_2)$$
Coz je presne jako leva strana, mame hotovo.

\subsection*{d)$\overline{z_1\cdot z_2} = \overline{z_1}\cdot \overline{z_2}$}

Postup podobny jako vyse, pro levou stranu:
$$\overline{(x_1+jy_1)\cdot(x_2+jy_2)} = \overline{x_1 x_2 + jx_1 y_2 +jx_2 y_1+y_1 y_2} = \overline{(x_1 x_2 +y_1 y_2) + j (x_1 y_2+ x_2 y_1)} = $$ 
$$ = (x_1 x_2 +y_1 y_2) - j (x_1 y_2+ x_2 y_1)$$
Pro pravou stranu:
$$\overline{x_1+j y_1}\cdot \overline{x_2+j y_2} = (x_1 - jy_1)\cdot (x_2 - jy_2) = x_1 x_2 - jx_1 y_2 -jx_2 y_1+y_1 y_2= (x_1 x_2 +y_1 y_2) - j (x_1 y_2+ x_2 y_1)$$
Coz presne odpovida leve strane.

\subsection*{e) $|z_1+z_2| \leq |z_1| + |z_2|$}
Tentokrat to budu resit jako rovnici, tedy obe strany naraz. Kdybych to resila zvlast, moc daleko bych se nedostala.(samozrejme jsem to zkusila nejprve resit zvlast a narazila :) ) Kdyz to budeme resit dohromady, veci se nam vykrati.
$$|(x_1+jy_1 + x_2 jy_2| \leq |x_1+jy_1| + |x_2 + jy_2|$$
$$\sqrt{(x_1+x_2)^2+(y_1+y_2)^2} \leq \sqrt{x_1^2 +y_1^2}+\sqrt{x_2^2 + y_2^2}$$
Chtela bych se zbavit tech odmocnin, tak to umocnim na druhou, pozor, prava strana se umocni dle vzorce $(a+b)^2 = a^2+2ab+b^2$.
$$(x_1+x_2)^2+(y_1+y_2)^2 \leq (x_1^2 +y_1^2) + 2 \cdot \sqrt{x_1^2 +y_1^2} \cdot \sqrt{x_2^2 + y_2^2} + (x_2^2 + y_2^2)$$
Na leve strane roznasobime co se da:
$$x_1^2+2x_1 x_2 + x_2^2 + y_1^2 + 2y_1 y_2 + y_2^2 \leq x_1^2 +y_1^2 + 2 \cdot \sqrt{x_1^2 +y_1^2} \cdot \sqrt{x_2^2 + y_2^2} + x_2^2 + y_2^2$$
Vykratime co se da:
$$2 x_1 x_2 + 2 y_1 y_2 \leq 2 \cdot \sqrt{x_1^2 +y_1^2} \cdot \sqrt{x_2^2 + y_2^2} $$
Toto muzu vydelit dvema, tzn vsechny dvojky zmizi:
$$x_1 x_2 + y_1 y_2 \leq \sqrt{x_1^2 +y_1^2} \cdot \sqrt{x_2^2 + y_2^2} $$
Protoze mezi odmocninami na prave strane je nasobeni, muzu to soupnout pod jednu odmocninu:
$$x_1 x_2 + y_1 y_2 \leq \sqrt{(x_1^2 +y_1^2) \cdot (x_2^2 + y_2^2)} $$
Ted se chci zbavit odmocniny na prave strane, zase to umocnim:
$$(x_1 x_2 + y_1 y_2)^2 \leq (x_1^2 +y_1^2) \cdot (x_2^2 + y_2^2) $$
Umocnim na leve strane, roznasobim na prave:
$$x_1^2 x_2^2 + 2\cdot x_1 x_2 y_1 y_2 + y_1^2 y_2^2 \leq x_1^2 x_2^2 +x_1^2 y_2^2 + x_2^2 y_1^2 + y_1^2 y_2^2$$
Opet pokratim co se da a ziskam:
$$ 2\cdot x_1 x_2 y_1 y_2 \leq x_1^2 y_2^2 + x_2^2 y_1^2$$
V tom ale porad nejak nevidim, zda to plati. Vidim v obou stranach ale cleny vzorce $(a-b)^2 = a^2-2ab+b^2$, prevedu teda levou stranu napravo napr:
$$ 0 \leq x_1^2 y_2^2 -2\cdot x_1 x_2 y_1 y_2 + x_2^2 y_1^2$$
$$ 0 \leq (x_1 y_2 - x_2 y_1)^2$$

Na teto rovnici uz muzeme vyrok posoudit. Bud budou cleny takove, ze se presne odectou, tim padem to bude na prave strane nula a bude platit rovnost. Nebo cleny komplexnich cisel budou jine, ale diky druhe mocnine cela prava strana bude vzdy kladna, tedy vetsi nez nula. Tedy dana nerovnost plati. Mame hotovo.

\section{Necht $z_1$, $z_2$, $z_3 \in \mathbb{C}$ tvori tri vrcholy rovnobezniku. Cemu se rovna cvtry vrchol $z_4$ protilehly k $z_2$?}

Rovnobeznik je bud ctverec, nebo obdelnik. Kdyz si nakreslite souradny system v kartezske rovine (osa x je tvorena realnou osou, osa y je imaginarni) a zde nakreslite obdelnik (muze byt i natoceny obecne) a do jeho rohu napisete patricna komplexni cisla (nejlepe za sebou v poradi, kde zacnete nezalezi). Ja jsem zacala napriklad v levem dolnim rohu. Kdyz zacnete jinak, patricne rovnice budou trochu jine, ale vysledek stejny. Pak $z_1, z_2$ tvori jednu zakladnu, $z_3, z_4$ druhou. Obe zakladny jsou stejne velke, oznacme usecku zakladny treba "a", z analyticke matematiky se urci jako rozdil konecnych bodu, tedy:
$$a=z_2 - z_1$$
$$a=z_3 - z_4$$
Z druhe rovnice vyjadrime $z_4$, to hledame a za "a" dosadime z prvni rovnice.
$$z_4 = -(a - z_3) = -(z_2-z_1 -z_3) = -z_2 + z_1 + z_3 $$
Preskupeni clenu jen pro krasu:
$$z_4 = z_1 - z_2 + z_3$$
Pozor, ve skriptech je vysledku chyba. Kdyz si udelate nejaky priklad, treba obdelnik rovnobezny s osami, bez natoceni a dosadite si zvlast hodnoty x, a zvlast hodnoty y bude vam toto reseni sedet.

\section{Urcete podminku na tri tuzne body $z_1, z_2, z_3 \in \mathbb{C}$, aby lezely na primce}

TODO: Vyplyva z Apolloinovy kruznice

\section{Necht $z_1, z_2, z_3 \in \mathbb{C}$ jsou takove, ze $z_1 +z_2 +z_3 = 0$, $|z_1|=|z_2|=|z_3| =1$. Ukazte, ze pak nutne tvori vrcholy rovnostranneho trojuhelnika vepsaneho do jednotkove kruznice.}

To, ze body lezi na jednotkove kruznici je jasne uz dano tim, ze jejich modul (velikost je jedna). Ted uz jen dokazat, ze tvori rovnostranny trojuhelnik Tady jde o obecnou vlastnost, neni dulezite konkretni ciselne reseni. Mam tri komplexni cisla, jedno z nich si tedy zvolim tak, aby velikost byla jednickova. Muzu si zvolit libovolne, nezalezi na konkretni ciselnem reseni. Tak proc si delat zivot slozitym, zvolim si:
$z_1 = 1$ a to dosadim do rovnice:
$$z_1 +z_2 +z_3 = 0$$
$$1 + (x_2 + j y_2) + (x_3 +jy_3) = 0$$
$$(x_2+x_3) +j(y_2 + y_3) = -1$$
Porovnanim leve a prave rovnice se nam reseni rozpadne na rovnici pro realnou a imaginarni cast zvlast:
$$x_2+x_3 = -1$$
Takze napriklad vyjadrim $x_2$ v zavislosti na $x_3$:
$$x_2 = -1 - x_3$$
Pro imaginarni slozku:
$$y_2+y_3 = 0$$
$$y_2 = - y_3$$

Pro dopocitani bodu vyuzijeme vlastnosti $|z_2| = 1$:
$$|z_2| = 1 = \sqrt{x_2^2+y_2^2} = \sqrt{(-1-x_3)^2 + (-y_3)^2} = \sqrt{1+2x_3 + x_3^2 + y_3^2}$$
Abychom se zbavili odmocniny, umocnime: ($1^2$ je porad 1 :) )
$$1 = 1+2x_3 + x_3^2 + y_3^2$$
Dale z vlastnosti $|z_3| = 1$ vime, ze $\sqrt{x_3^2 + y_3^2}=1$, tedy i $x_3^2 + y_3^2 = 1$, tuto posledni rovnici dosadime vyse a ziskame:
$$1 = 1+2x_3 + 1$$
$$-1 = 2x_3$$
$$x_3 = -\frac{1}{2}$$
Z toho muzeme dopocitat $y_3$:
$$x_3^2 + y_3^2 = 1$$
$$\frac{1}{4} + y_3^2 = 1$$
$$y_3 = \sqrt{\frac{3}{4}}$$
$$y_3 = \frac{\sqrt{3}}{2}$$
A z toho teda konecne $x_2$ a $y_2$:
$$x_2 = -1 - -\frac{1}{2} = -\frac{1}{2}$$
$$y_2 = -y_3 = -\frac{\sqrt{3}}{2}$$

Takze mame ted vsechny vrcholy, jeste bychom meli urcit, zda vzdalenost mezi nimi je stejna (tvori rovnostranny trojuhelnik:
$$a_{12} = z_2 - z_1 = -\frac{1}{2}-j\frac{\sqrt{3}}{2} - 1 = -\frac{3}{2}-j\frac{\sqrt{3}}{2}$$
$$|a_{12}| = \sqrt{\frac{9}{4}+\frac{3}{4}} = \sqrt{\frac{12}{4}} = 3$$
$$a_{23} = z_3 - z_2 = -\frac{1}{2} + j\frac{\sqrt{3}}{2} - \left( -\frac{1}{2} -j\frac{\sqrt{3}}{2}  \right) = 0+j2\frac{\sqrt{3}}{2}$$
$$|a_{23}| = \sqrt{0+4\frac{3}{4}} = 3$$
$$a_{31} = z_1 - z_3 = 1 - \left( -\frac{1}{2}+j\frac{\sqrt{3}}{2} \right) = \frac{3}{2} - j\frac{\sqrt{3}}{2}$$
$$|a_{31}| = \sqrt{\frac{9}{4}+\frac{3}{4}} = 3$$

Vsechny tri strany maji stejnou velikost, tedy jsme dokazali, ze trojuhelnik je rovnostranny.

\section{Popiste nasledujici mnoziny:}
%TODO: vsechny priklady + obrazky
\subsection*{a) $M = \{ z \in \mathbb{C} \vert |z-j| \leq 1, Re z > 0$}
Tyto priklady se resi nacrtem. Vzdy je dobre zacit osami (x realne, y imaginarni), jako kriz. Tedy i myslet na zapornou cast. Rovnice typu:
$$|z-a|\leq b$$
vyjadruji vzdy! rovnici. To je dobre si zapamatovat, bude se hodit i pozdeji. Bod $a$ udava stred kruznice, tedy v nasem pripade se nachazi na ose y v bode $1j$. Cislo b je pak polomer, v nasem pripadne 1. Kruznice tedy prochazi bodem $2j, 1+1j, 0, -1+1j$

Druha podminka rika, ze $Re z > 0$, tedy jen takova komplexni cisla, jejich realna hodnota je vetsi nez nula... To znamena, ze se jedna o pravou cast od imaginarni osy (zdurazneni, tedy jak nad osou x, tak pod osou x.

Pokud oba pozadavky spojime, udelame prunik, pak mnozina, kterou hledam je pouze prava polokruznice, tedy kruh prochazejici body: $2j, 1+1j, 0$ a zpet do bodu $2j$.

\subsection*{b)}
Zde je mezikruzi, rovnice $1 < |z| < 3$ se vyresi zvlast na dva pripady:
$1<|z|$ a $|z|<3$. Arg z znamena uhel komplexniho cisla...

\section{Pro ktera $z \in \mathbb{C}$ plati:}
\subsection*{$\operatorname{Re} \frac{1}{z} = \alpha$, $\alpha \in R$}

Zde mame nalezt komplexni cislo z, tedy jeho realnou $x$ a imaginarni slozku $y$. Nejprve si upravime vyraz:
$$\frac{1}{z} = \frac{1}{x+jy} = \frac{1}{x+jy}\cdot \frac{x-jy}{x-jy} = \frac{x-jy}{x^2+y^2}$$
Pouzili jsme jiz znamy trik usmerneni. Ze zadani nas zajima pouze realna slozka, tedy:
$$\frac{x}{x^2+y^2} = \alpha$$
Roznasobime:
$$x = \alpha x^2 + \alpha y^2$$
V tom muzeme videt jistou podobnost s vyjadrenim kruhu v analyticke matematice: $(x-a)^2 + (y-b)^2 = r^2$, budeme se to tedy do teto podoby snazit dostat:
$$0 = \alpha x^2 - x + \alpha y^2$$
Vytknu $\alpha$ protoze potrebuju jen samotne $x^2$ v zavorce:
$$0 = \alpha (x^2 - \frac{x}{\alpha})+\alpha y^2$$
Ted zavorku potrebuji dostat $(a-b)^2$, zde na to pouziji trik, co se ucil uz na stredni ci zakladce. Ve vyrazu $x^2 - \frac{x}{\alpha}$ vidim cast vzorce $a^2-2ab+b^2$. Konkretne pozoruji, ze clen $b^2$ chybi. Tak si ho tam pridam. Ale nemuzu jen tak neco pridat... Co pridam, musim zase odecist (abych ve skutecnosti pridala 0). Jak zjistim, co je nutno pridat? Vim, ze prostredni clen odpovida:
$$-\frac{x}{\alpha} = -2ab$$
Vim, ze clen $a = x$, hledam clen b:
$$b = \frac{1}{2\alpha}$$
$$b^2 = \frac{1}{4 \alpha^2}$$
Clen $b^2$ tedy pridam do rovnice, nejprve ji ale zopakuji pro prehlednost, az v dalsim radku pridam vyraz dvakrat -jednou prictu a podruhe odectu.
$$0 = \alpha (x^2 - \frac{x}{\alpha})+\alpha y^2$$
$$0 = \alpha  \left(x^2 - \frac{x}{\alpha} + \frac{1}{4 \alpha^2} - \frac{1}{4 \alpha^2}\right) +\alpha y^2$$
Posledni vyraz (co odecitam) vytknu ze zavorky (abych tam mela jen cleny $a^2-2ab+b^2$, pozor na to, ze pred zavorkou je $\alpha$, tedy se roznasobi posledni clen, kdyz jde ze zavorky pryc:
$$0 = \alpha \left(x^2 - \frac{x}{\alpha} + \frac{1}{4 \alpha^2}\right)- \frac{1}{4 \alpha} +\alpha y^2$$
Ted tedy od vzorce  $a^2-2ab+b^2$ muzeme prejit na $(a-b)^2$:
$$0 = \alpha \left(x-\frac{1}{2\alpha}\right)^2  - \frac{1}{4 \alpha} +\alpha y^2$$
Clen $ - \frac{1}{4 \alpha}$ neobsahuje ani jednu promenou, je to cislo, presuneme ho tedy na druhou stranu:
$$ \frac{1}{4 \alpha} = \alpha \left(x-\frac{1}{2\alpha}\right)^2 +\alpha y^2$$
Z definice kruznice pred cleny $x^2, y^2$ nema byt zadny clen, tedy rovnici vydelim $\alpha$:
$$ \frac{1}{4 \alpha^2} = \left(x-\frac{1}{2\alpha}\right)^2 +y^2$$
A toto je rovnice kruznice. Jeji parametry zjistim nasledovne. Na prave strane je cislo odpovidajici $r^2$, tedy:
$$r^2 = \frac{1}{4\alpha^2}$$
$$r = \frac{1}{2\alpha}$$
Stred kruznice je dan hodnotou, pro kterou se cleny
$\left(x-\frac{1}{2\alpha}\right)$ a $y^2$ (obecne muze byt take zavorka a slozitejsi vyraz) rovnaji nule, tedy:
$$x-\frac{1}{2\alpha} = 0$$
$$x = \frac{1}{2\alpha}$$
$$y = 0$$

Tedy zaver je, ze rovnice ze zadani plati pro vsechna $z \in \mathbb{C}$, ktera lezi na kruznici (pouze na oblouku, ne uvnitr!) se stredem $S=\left[\frac{1}{2\alpha},0\right]$ a polomerem $r=\frac{1}{2\alpha}$

\subsection*{b) }

\section{Jaky je obraz cisel $1, -1, j, \frac{1-j}{\sqrt{2}}$ pri stereograficke projekci?}
Stereograficka projekce cisla $z=x+jy$ je dana vzorcem:
$$\Phi = \left[ \frac{x}{x^2+y^2+1}; \frac{y}{x^2+y^2+1}; \frac{x^2+y^2}{x^2+y^2+1}\right] $$

\subsection*{a) $z=1$}
Tedy $x=1, y=0$. To dosadime do zmineneho vzorce a ziskame:
$$\Phi = \left[ \frac{1}{2};0; \frac{1}{2}\right]$$

\subsection*{b) $z=-1$}
Tedy $x=-1, y=0$. To dosadime a ziskame:
$$\Phi = \left[\frac{-1}{2};0;\frac{1}{2}\right]$$

\subsection*{c) $z=j$}
Tedy $x=0, y=1$:
$$\Phi = \left[0;\frac{1}{2};\frac{1}{2}\right]$$

\subsection*{d) $z= \frac{1-j}{\sqrt{2}}$}
Tedy $x=\frac{1}{\sqrt{2}}, y= -\frac{1}{\sqrt{2}}$
$$\Phi = \left[ \frac{\frac{1}{\sqrt{2}}}{\frac{1}{2}+\frac{1}{2}+1}; \frac{-\frac{1}{\sqrt{2}}}{\frac{1}{2}+\frac{1}{2}+1};\frac{\frac{1}{2}+\frac{1}{2}}{\frac{1}{2}+\frac{1}{2}+1}\right]$$
$$\Phi = \left[ \frac{1}{2\sqrt{2}};-\frac{1}{2\sqrt{2}};\frac{1}{2}\right]$$




