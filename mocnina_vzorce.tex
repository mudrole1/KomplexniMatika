\chapter{Reprezentace holomorfni funkce mocninou radou}

\section{Motivace}
%TODO

\section{Vzorce}
Mocnina rada se stredem v bode $z_0$:
$$\sum_{n=0}^\infty a_n (z-z_0)^n = a_0 + a_1 (z-z_0) +a_2 (z-z_0)^2 + \dots$$
Tedy se jedna o sumu ruznych polynomialnich funkci s promenou z a koeficienty $a_n$.

\subsection{Typy konvergence}
\subsubsection*{Bodova}
\begin{equation}
\label{eq:bod_kon}
Sm(z)-f(z)| \to 0 \; \operatorname{pro} \; n \to \infty
\end{equation} 

\subsubsection*{Stejnosmerna}
\begin{equation}
\label{eq:ste_kon}
\operatorname{sup}_{z \in M} |Sm(z)-f(z)| \to 0 \; \operatorname{pro} \; n \to \infty
\end{equation}

\subsection{Kriteria pro urceni konvergence a jejiho polomeru $R$}

\subsubsection*{Odmocninove}
\begin{equation}
\label{eq:odm}
\frac{1}{R} = \lim_{n \to \infty} \sqrt[n]{a_n}
\end{equation}
\subsubsection*{Podilove}
\begin{equation}
\label{eq:pod}
\frac{1}{R} = \lim_{n \to \infty} \frac{|a_{n+1}|}{|a_n|}
\end{equation}
\subsubsection*{Srovnavaci}
%TODO: doplnit
Rada konverguje pokud $\frac{1}{R} < 1$

\subsection*{Soucet rady}
Pro specialni pripad rady (tzv. geometricka) $\sum_{n_0}^\infty z^n$ vime, ze jeji soucet se da vypocitat jako:
\begin{equation}
\label{eq:soucetr}
S(z) = \frac{1}{1-z}
\end{equation}
Tedy, kdyz jsme dotazani na vypocet souctu rady, snazime se pouzit figle na to, jak tam najit podobnost prave s geometrickou radou. Tim padem pak muzeme pouzit tento vzorec. Figle jsou dva:
\begin{itemize}
\item Derivace rady - ale pozor, musime pak zintegrovat zpet soucet!
\item Prechod k primitivni funkci (integrace) - ale pozor, musime pak zderivovat soucet!
\end{itemize}

\subsection*{\label{sec:rozviteFce} Uzitecne vzorecky}
$$e^z = \sum_{n=0}^\infty \frac{z^n}{n!}$$
$$sin(z) = \sum_{n=0}^\infty (-1)^n \frac{z^{(2n+1)}}{(2n+1)!}$$
$$cos(z) = \sum_{n=0}^\infty (-1)^n \frac{z^{(2n)}}{(2n)!} $$
$$ln(z) = \sum_{n=0}^\infty \frac{(-1)^{(n-1)}}{n}(z-1)^n$$
$$arctg(z) = \sum_{n=0}^\infty (-1)^n \frac{z^{(2n+1)}}{2n+1}$$


