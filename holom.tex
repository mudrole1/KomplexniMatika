\newpage

\section{Vypoctete integraly podle zadanych krivek}
\subsection{a) $\int_C Re(z), \; C=\{z \in \mathbb{C}, |z-z_0|=r\}$, kladne orientovana}

Takto zadana krivka je kruh, muzeme ho tedy prepsat do jineho tvaru a to:
$$\varphi(t) = z_0 + r\cdot e^{jt} = z_0 + r(\operatorname{cos}(t)+j\operatorname{sin}(t))$$
pro $t \in (0;2\pi)$

Kdyz pocitame krivkovy integral, tak posupujeme tak, ze krivku $\varphi(t)$ dosadime za promenou $z$ v integralu a nasobime derivaci krivky podle $t$, neboli:

$$\int_C f(z) = f(\varphi(t))\cdot \varphi ' (t)$$

Tedy v nasem pripade: (pozor, funkce je jen realna cast, tedy dosadim jen realnou cast krivky, tedy se mi hodi, ze jsem si to rozlozila na goniometricky tvar a navic, i bod $z_0$ ma realnou a imaginarni cast):
$$\int_C Re(z) = \int_0^{2\pi} Re(z_0+r(\operatorname{cos}(t)+j\operatorname{sin}(t)))\cdot rje^{jt} dt = \int_0^{2\pi} (x_0+r\operatorname{cos}(t)) \cdot rje^{jt} dt =$$
Zavorku roznasobim:
$$=\int_0^{2\pi} x_0rje^{jt} + r\operatorname{cos}t\cdot rje^{jt} dt=$$
A ted to roztrhnu na dva integraly, abych je vubec mohla spocitat:
$$=\int_0^{2\pi}x_0rje^{jt} dt + \int_0^{2\pi} r^2\operatorname{cos}tje^{jt} dt= $$
V prvnim integralu je pouze funkce $e^{jt}$ zavisla na promene $t$, podle ktere integrujeme. Tedy muzeme vse ostatni vytknout pred integral. V druhem mame funkci $\operatorname{cos}t$ a $e^{jt}$, tedy neumime zintegrovat dve funkce naraz. Tady muzeme zkusit pouzit trik, kdy fci rozepiseme: $e^{jt} = \operatorname{cos}t + j\operatorname{sin}t$. Tedy vse dohromady:
$$=x_0 rj\int_0^{2\pi}e^{jt} dt + r^2 j\int_0^{2\pi} \operatorname{cos}t\cdot (\operatorname{cos}t + j\operatorname{sin}t)dt = $$
Prvni integral uz muzeme zintegrovat (je to klasicke integrovani, co se delalo v prvaku. Je dobre mit na pameti nejake zakladni funkce. Napr fce $e^t$ je po integrovani zase $e^t$, tady je problem ale, ze v exponentu je vice veci, musime udelat substituci:
$$jt = u$$
$$jdt = du$$
$$dt = \frac{du}{j}$$
(a musime take do substituce dosadit meze a prepocitat)
Druhy integral pak roznasobime zavorku:
$$ = x_0rj \int_0^{2\pi j} e^u \frac{du}{j}+ r^2 j\int_0^{2\pi} \operatorname{cos}^2t + j\operatorname{cos}t\operatorname{sin}tdt = $$
Prvni integral zintegrujeme a druhy muzeme rozseknout opet na dva:
$$= x_0rj \frac{1}{j} \left[ e^u\right]_0^{2\pi j} + r^2 j\int_0^{2\pi} \operatorname{cos}^2t dt + r^2 j\int_0^{2\pi} j\operatorname{cos}t \operatorname{sin}tdt = $$
Do prvniho clenu dosadime meze (nejprve horni minus dolni), druhy integral zatim preskocim, treti se resi opet substituci (typicky priklad na kombinaci kosinu a sinu!), tedy napr:
$$\operatorname{sin}t = u$$
$$\operatorname{cos}t dt = du$$
$$dt = \frac{du}{\operatorname{cos}t}$$
A prepocet mezi:
$$\operatorname{sin}(0) = 0$$
$$\operatorname{sin}(2\pi) = 0$$
Ten posledni krok (deleni) matematici nemaji moc radi, tak si ho napiste nekam bokem, a pak zacmarejte, ale jinak vas k vysledku dobre navede. Tedy opet vse dohromady:
$$ = x_0 r (e^{2\pi j} - e^0) + r^2 j\int_0^{2\pi} \operatorname{cos}^2t dt + r^2 j^2\int_0^{0} u du = $$
V prnim clenu vidime vztah $e^{2\pi j}$, coz nam muze pripominat \textit{eulerovu identitu} (viz vypsane vzorce), jen to musime malicko upravit. Figl je zalozen na tom, ze nasobeni exponentu je mocneni zakladu. V tretim clenu se zbavime $j^2 = -1$ a taky zintegrujeme, vse dohromady:
$$ = x_0 r ( \left(e^{\pi j}\right)^2 - 1) + r^2 j\int_0^{2\pi} \operatorname{cos}^2t dt - r^2 \left[ \frac{u^2}{2}\right]_0^0 = $$
V prvnim clenu uz vidime eulerovu identitu, tedy vime, ze $e^{\pi j} = -1$, do tretiho dosadime meze:
$$ = x_0 r ( (-1)^2 -1 )+ r^2 j\int_0^{2\pi} \operatorname{cos}^2t dt - r^2 \left[ 0 -0\right] = $$
Pozoruji, ze v prvnim clenu v zavorce vznikne nula, i v tretim clenu je v zavorce nula. Tedy prvni i treti clen jsou nulove a zbyde pouze druhy clen:
$$= r^2 j\int_0^{2\pi} \operatorname{cos}^2t dt = $$
Integrovat $\operatorname{cos}^2t $ neni snadne. Nelze to resit ani subsitutci (po derivaci bychom dostali sinus, ktery v integralu neni) a ani metodou perpartes (opet bychom si tam zanesli sinus). Muzeme to ale resit dvema rozklady. Bud mame nekde na tahaku vypsane uzitecne vzorce pro praci s goniometrickym funkcemi, tedy vime, ze $\operatorname{cos}^2t = \frac{1-\operatorname{cos}(2t)}{2}$ a nebo si to rozlozime na exponencialni funkce. Ukazu druhy postup:
$$ = r^2 j \int_0^{2\pi} \left( \frac{e^{jt} + e^{-jt}}{2} \right) ^2 dt = $$
To umocnime, pozor! Citatel musime mocnit podle vzorce $(a+b)^2 = a^2+2ab+b^2$ a rovnou roztrhnu na tri integraly:
$$ = r^2 j \frac{1}{4} \int_0^{2\pi} e^{j2t} dt + r^2 j \frac{1}{4} \int_0^{2\pi} 2e^{jt}e^{-jt} dt + r^2 j \frac{1}{4} \int_0^{2\pi} e^{-j2t} dt =$$
Exponencialni funkci umime pekne itegrovat s vyuzitim substituce, tedy pro prvni integral:
$$j2t = u$$
$$j2 dt = du$$
$$dt = \frac{du}{2j}$$
Pro treti integral:
$$-j2t = v$$
$$-j2 dt = dv$$
$$dt = -\frac{dv}{2j}$$
Prepoctu i meze a dosadim. Navic v druhem integralu prevedu nasobeni zakladu exponencilani fce na soucet exponentu.
$$ =  r^2 j \frac{1}{4} \int_0^{4\pi j} e^u \frac{du}{2j} + r^2 j \frac{1}{2} \int_0^{2\pi} e^{jt-jt} dt +  r^2 j \frac{1}{4} \int_0^{-4\pi j} e^v \left( -\frac{dv}{2j} \right) = $$
jmenovatele $2j$ u prvniho a tretiho integralu muzu vytknout a jinak zintegruji, funkce $e^{neco}$ zustane opet $e^{neco}$, tedy: (pozor, ze se pred tretim integralem zmenilo znamenko na minus, protoze jsem ho tam vyktla ze zavorky)
$$ = r^2 j \frac{1}{4} \frac{1}{2j} \left[ e^u \right]_0^{4\pi j} +  r^2 j \frac{1}{2} \int_0^{2\pi} e^0 dt - r^2 j \frac{1}{4} \frac{1}{2j} \left[ e^u \right]_0^{-4\pi j} = $$
Pro 1. a 3. integral pokratim a dosadim meze, opet vidim, ze to pujde na eulerovu identitu, tak uz to zacnu smerovat na trik jako pred chvili. U druheho vidim, ze $e^0 =1$
$$ = r^2 \frac{1}{8} \left( (e^{\pi j})^4 - e^0 \right) +  r^2 j \frac{1}{2} \int_0^{2\pi} 1 dt - r^2 \frac{1}{8} \left( (e^{\pi j})^{-4} - e^0 \right) =$$
Cely prvni clen bude nulovy, protoze $e^{\pi j} = -1$, a $(-1)^4 = 1$, tedy se odecte s $e^0 = 1$. Treti clen jeste pro nazornost rozepisu, ale uz take pouziji eulerovu identitu:
$$ = 0 +  r^2 j \frac{1}{2} \int_0^{2\pi} 1 dt - r^2 \frac{1}{8} \left( \frac{1}{(-1)^4} - 1 \right)  = $$
Pozorujeme, ze i treti integral bude nulovy. Zustal jen prostredni, kdy mohu integrovat:
$$ = r^2 j \frac{1}{2} \left[ t \right]_0^{2\pi} = r^2 j \frac{1}{2} (2\pi - 0) = r^2 j \pi$$

Uff a mam vysledek :) Ve skutecnosti to neni tak dlouhe, jen ja to tu rozpisuji hodne dopodrobna. Na papire to muzete mit klidne i na 5 radek jako ja ve svem vypoctu :)

\newpage

\section{Pro ktera $z \in \mathbb{C}$ plati, ze}
\subsection{a) $\int_0^1 \operatorname{sin}(tz) dt = 0$}
Integral muzeme vyresit s vyuzitim substituce:
$$tz = u$$
$$z dt = du$$
$$dt = \frac{du}{z}$$
+ prepocteme meze:
$$0\cdot z = 0$$
$$1\cdot z = z$$
Tedy:
$$\int_0^1 \operatorname{sin}(tz) dt = \int_0^z \operatorname{sin}(u) \frac{1}{z} du = \left[ \frac{-\operatorname{cos}(u)}{z} \right]_0^z = \frac{-\operatorname{cos}(z)}{z} -\left( \frac{-\operatorname{cos}(0)}{z} \right)= \frac{-\operatorname{cos}(z)}{z} + \frac{1}{z} = \frac{-\operatorname{cos}z+1}{z}$$
A takto upravena leva strana se ma tedy rovnat nule, tedy resime:
$$\frac{-\operatorname{cos}z+1}{z} = 0$$
Aby zlomek byl nula, musi byt jedine citatel nulovy, jmenovatel to neovlivni. Tedy resime:
$$-\operatorname{cos}z+1 = 0$$
$$-\operatorname{cos}z = -1$$
$$\operatorname{cos}z = 1$$

Funkce kosinus je rovna hodnota 1 pro hodnoty $z = 0, 2\pi, 4\pi \dots$, tedy to muzeme zapsat jako:
$$z = 2k\pi,\; k \in \mathbb{Z}$$

\subsection{b) komplikovane}

\newpage

\section{Necht $C$ je jednoducha uzavrena kladne orientovana krivka neprochazejici bodem $z_0 \in \mathbb{C}$. Zjistete, jakych moznych hodnot muze nabyvat integral $\int_c (z-z_0)^n$ v zavislosti na $n \in \mathbb{Z}$ a poloze bodu $z_0$ vuci krivce.}

Pro vyjadreni v zavislosti na parametru $n$ pouziji Cauchyho vzorec \ref{eq:cauchyho}:
$$\int_C \frac{f(z)}{(z-z_0)} = 2\pi j f(z_0)$$

Abych ho mohla pouzit spravne, musim si zadanou funkci ale prevest do stejneho tvaru, tedy jen vytknu jeden clen do jmenovatele a exponent se mi tedy zvetsi:

$$\int_c (z-z_0)^n = \int_C \frac{(z-z_0)^{n+1}}{(z-z_0)}$$

Budeme vychazet z Cauchyho vzorce, nutno podotknout, ze novou funkci $f(z) = (z-z_0)^{n+1}$. Muzu tedy ted vyresit 2 pripady:
\begin{itemize}
\item $n+1 \neq 0$, tedy $n < -1$, z vzorce plyne, ze $2\pi j (z_0 - z_0)^{n+1}$ (dosadila jsem hodnotu $z_0$ za promenou $z$ do funkce $f(z)$. Tedy vidim, ze se to rovna nule. Tedy  $\int_c (z-z_0)^n = 0$ pro $n \neq -1$
\item $n+1 = 0$, tedy $n = -1$, zde se funkce $f(z)$ zjednodusi na $f(z)=(z-z_0)^0 = 1$, neni zde zadna promena $z$, kam bychom dosadili, tedy s vyuziti cauchyho vzorce nam vyjde pouze $2\pi j$, tedy shrnuti:
 $\int_c (z-z_0)^n = 2\pi j$ pro $n= -1$
\end{itemize}

Zhlediska polohy bodu $z_0$ rozeznavame dva pripady:
\begin{itemize}
\item $z_0$ lezi vne krivky - z Cauchyho vety plyne, ze integral je nulovy. Tato vlastnost je dulezita! Mit ji na pameti/ tahaku! Ale pozor, jen pro pripad, kdy $n \neq 0$! To je specialni pripad.
\item $z_0$ lezi uvnitr krivky - zalezi na hodnote $n$, viz reseni vyse. 
\end{itemize}

\newpage

\section{Necht $C$ je uzavrena jednoducha kladne orientovana krivka. Vypoctete hodnoty integralu $\int_C \frac{1}{z_2+9}$}

Funkci musime nejprve rozlozit, abychom videli koreny, tedy: (koreny najdeme polozenim jmenovatele rovno nule)
$$\int_C \frac{1}{(z+3j)(z-3j)}$$
Vidime, ze funkce ma dva poly, musime tedy resit 4 pripady:
\begin{enumerate}
\item Krivka $C$ ma bod $z_0=3j$ uvnitr
\item Krivka $C$ ma bod $z_0=-3j$ uvnitr
\item Krivka $C$ nema ani jeden z polu uvnitr
\item Krivka $C$ ma oba poly uvnitr
\end{enumerate}
Opet pouzijeme Cauchyho vzorec, \ref{eq:cauchyho}. 

\subsection*{Za 1.: $z_0=3j$ uvnitr krivky.}

Pointa je, ze funkce $f(z)$ musime prevest do tvaru $\frac{f_n(z)}{(z-z_0)}$, tedy v nasem pripade: $\frac{f_n(z)}{(z-3j)}$, kde $f_n(z)$ oznacuje novou funkci. Vysledek pak bude $2\pi j f_n(z_0)$.
Tedy:
$$\int_C \frac{1}{(z+3j)(z-3j)} = \int_C \frac{\frac{1}{z+3j}}{z-3j} = 2 \pi j \frac{1}{z_0 +3j} = 2 \pi j \frac{1}{3j+3j} = 2 \pi j \frac{1}{6j} = \frac{\pi}{3}$$

\subsection*{Za 2.: $z_0 = -3j$ uvnitr krivky.}
Postupuje stejne, jako v prvnim bode:
$$\int_C \frac{1}{(z+3j)(z-3j)} = \int_C \frac{\frac{1}{z-3j}}{z-(-3j)} = 2\pi j \frac{1}{-3j -3j} = 2\pi j \frac{1}{-6j} = -\frac{\pi}{3}$$

\subsection*{Za 3.: zadny pol nelezeni uvnitr krivky.}
Tedy $z_0$ lezi vne krivky, pouzijeme Cauchyho vetu, z ktere plyne, ze takovy integral je nulovy, tedy zapsano:
$$\int_C \frac{1}{(z+3j)(z-3j)} = 0 $$

\subsection*{Za 4.: oba poly lezi uvnitr krivky.}
Cauchyho vzorec je pouze pro jeden pol $z_0$, musime tedy funkci roztrhnout na dve funkce se svymi jednonasobnymi poly. K tomu nam skvele poslouzi parcialni zlomky:
$$\frac{1}{(z+3j)(z-3j)} = \frac{A}{z+3j} + \frac{B}{z-3j}$$
Ve jmenovateli jsou patricne koreny a citatele musime dopocitat, aby rovnost platila. Nutno podotknout, ze citatel ma vzdy o stupen mene nez jemnovatel. Tedy v nasem konkretnim pripade je menovatel stupne 1 (promena $z$ je jen na prvni), tedy v citateli staci konstanta. Kdyby v jmenovateli bylo $z^2$, citatel by musel mit tvar $Az+B$.

Vyjdeme ze znalosti, ze pro soucet dvou zlomku plati:
$$\frac{a}{b}+\frac{c}{d} = \frac{ad+bc}{bd}$$
tedy v nasem pripade musi platit:
$$\frac{A(z-3j)+B(z+3j)}{(z+3j)(z-3j)} = \frac{1}{(z+3j)(z-3j)}$$
Porovnavame citatele:
$$Az -3jA + Bz + 3jB = 1$$
Porovname patricne stupne dle promene $z$:

Pro z: $A+B = 0$

Pro konstanty: $-3jA + 3jB = 1$

Z prvni rovnice dosadim do druhe:
$$-3jA + 3j(-A) = 1$$
$$-6jA = 1$$
$$A = -\frac{1}{6j}$$
$$B = -A = \frac{1}{6j}$$

Tedy jsme ziskali, ze:

$$\frac{1}{(z+3j)(z-3j)} = \frac{-\frac{1}{6j}}{z+3j} + \frac{\frac{1}{6j}}{z-3j}$$

A tedy krivkovy integral muzeme roztrhnout na dva:
$$\int_C \frac{1}{(z+3j)(z-3j)}  = \int_C \frac{-\frac{1}{6j}}{z+3j} + \int_C \frac{\frac{1}{6j}}{z-3j}$$

A pouzijeme opet Cauchyho vzorec. Je potreba ale dat pozor, co je bod $z_0$ pro dany integral. Pro prvni bychom ve jmenovateli meli mit tvar $z-z_0$, tedy vidime, ze to musime upravit $z+3j = z - (-3j)$. Pro prvni integral je tedy $z_0 = -3j$, pro druhy $z_0 =3j$. Ovsem funkce, kam bychom meli dosadit je jen $\frac{1}{6j}$, tedy neni kam bod $z_0$ dosadit. (protoze ho dosazujeme za promenou z) Tedy:

$$\int_C \frac{1}{(z+3j)(z-3j)} = 2\pi j\left( -\frac{1}{6j}\right) + 2\pi j \frac{1}{6j} = 0$$

\newpage

\section{Necht P(z) je polynom $P(z) = (z-z_1)(z-z_2)\dots(z-z_n)$, kde $z_1, z_2 \dots z_n$ jsou navzajem ruzna cisla. Jaky je maximalni mozny pocet ruznych hodnot integralu $\int_C \frac{1}{P(z)}$?}

Muzeme na toto jit postupne:
\begin{itemize}
\item n=1: mame moznost, ze bod $z_1$ lezi uvnitr nebo vne krivky, tedy mame 2 ruzne hodnoty.
\item n=2: Tuto moznost jsme videli na predchozim priklade. Prakticky jsou 4 kombinace, ale dulezite je zde, ze se nas ptaji na \textbf{ruzne} hodnoty. Zde si uplne nejsem jista oduvodnenim v teorii (TODO: zde by to chtelo rozsirit), ale vec, se ma tak, ze kdyz je funkce zadana ve tvaru $\frac{K}{f_n(z)}$, tedy v citateli je nejaka konstanta $k$ a pouze jmenovatel obsahuje promenou $z$, tak pripad, kdy krivka obsahuje vsechny poly, vyjde vzdy \underline{nulovy}, tedy stejna hodnota jako pro pripad, kdy krivka neobsahuje zadny pol. Tedy zde jsou jen maximalne 3 ruzne hodnoty.
\item n=3: Obdobne jako predchozi pripad. Je moznych 8 kombinaci, ale pripady, kdy krivka neobsahuje ani jeden pol, nebo vsechny, jsou nulove, tedy nejsou ruzne a mame jen 7 kombinaci.
\end{itemize}

Vysledek je tedy, ze:
\begin{itemize}
\item n=1: $2$ pripady
\item n>1: $2^{n-1}$ pripadu
\end{itemize}

\newpage

\section{Necht $C$ je uzavrena jednoducha kladne orientovana krivka neprochazejici body $\pm ja, \; a>0$. Zjistete vsechny hodnoty integralu $\int_C \frac{e^z}{(z^+a^2}$}

Tady postupujeme opet jako u prikladu 4. Nejrpve funkci rozlozime, abychom videli poly:
$$\int_C \frac{e^z}{(z^+a^2} = \int_C \frac{e^z}{(z+ja)(z-ja)}$$

Mame dva poly, budeme resit 4 pripady:
\begin{itemize}
\item Oba koreny vne krivky: Z cauchyho vety plyne, ze integral je nulovy.
\item $z_0 = ja$ uvnitr krivky sam: Postup jako u prikladu 4., upravim:
$$\int_C \frac{e^z}{(z+ja)(z-ja)} = \int_C \frac{\frac{e^z}{(z+ja)}}{z-ja} = 2\pi j \frac{e^{z_0}}{z_0+ja} = 2\pi j \frac{e^{ja}}{ja+ja} = \frac{\pi e^{ja}}{a}$$
\item $z_0 = -ja$ uvnitr krivky sam: 
$$\int_C \frac{\frac{e^z}{(z-ja)}}{z-(-ja)} = 2\pi j \frac{e^{-ja}}{-ja-ja} = -\frac{\pi e^{-ja}}{a}$$
Zde je mala chybka ve vysledkach ve skriptech.
\item Oba poly uvnitr krivky. Zde to neni primitivni (jako u prikladu 5), v citateli neni konstanta! Resime rozkladem na parcialni zlomky, jen funkce $e^z$ nas tam stve a motala by se v parcialnich zlomkach, a tak budeme resit rozklad $\frac{1}{(z+ja)(z-ja)}\cdot e^z$, tedy exponencialni fci dame bokem, a pak kazdy parcialni zlomek tim prenasobime.

Tedy resime:
$$\frac{1}{(z+ja)(z-ja)} = \frac{A}{z+ja} + \frac{B}{z-ja}$$
Pro nalezeni $A, B$ muzeme postupovat jako u prikladu 4, a nebo zkracene a to tak: Pro nalezeni konstanty $A$ je problemovy bod (koren jmenovatele) $-ja$. Tedy si v puvodnim zlomku zakryjte prstem celou zavorku $(z+ja)$ a do zbytku za $z$ dosadte prave $-ja$. Dostanete, ze:
$$A = \frac{1}{-2ja}$$
Obdobne, pro $B$ je problemovy bod $ja$, tedy si zakryjte zavorku $(z-ja)$ puvodniho zlomku a dosadte, dostanete, ze:
$$B = \frac{1}{2ja}$$
Ve vysledku jsme ziskali, ze:
$$\frac{e^z}{(z+ja)(z-ja)} = \left( \frac{\frac{1}{-2ja}}{z+ja} + \frac{\frac{1}{2ja}}{z-ja} \right) \cdot e^z = \frac{\frac{e^z}{-2ja}}{z+ja} + \frac{\frac{e^z}{2ja}}{z-ja} $$

A tedy krivkovy integral pro takto upravenou funkci muzeme resit na dvakrat:
$$\int_C  \frac{\frac{e^z}{-2ja}}{z+ja} + \int_C \frac{\frac{e^z}{2ja}}{z-ja} = 2\pi j \frac{e^{-ja}}{-2ja} + 2\pi j \frac{e^{ja}}{2ja} = -\frac{\pi e^{-ja}}{a}+ \frac{\pi e^{ja}}{a} =$$
Vytknu a prohodim poradi, protoze v tom vidim jistou podobnost s rozkladem funkce sinus ($\operatorname{sin}z = \frac{e^{jz} - e^{-jz}}{2j}$).
 $$ = \frac{\pi}{a} (e^{ja}-e^{-ja})$$
 Jen pro podobnost s fci sinus mi tam schazi $2j$, ale muzu tam dopsat zlomek $\frac{2j}{2j}$, coz je vlastne jednicka, takze tam nic nepridam. Tohle je jeden z matematickych tricku.
 $$ = \frac{\pi}{a} (e^{ja}-e^{-ja}) \cdot \frac{2j}{2j}$$
 A ted to uz muzu prevest na sinus:
 $$ = \frac{\pi 2j}{a} \operatorname{sin}(a)$$
\end{itemize}
A mame hotovo.

\newpage

\section{Necht $f(z) = \int_{-1}^1 \frac{1}{t-z} dt$ pro $z \in \mathbb{C} \backslash [-1,1]$.}

Nejprve vypocteme funkci a to nasledovne. Integral budeme resit substituci:
$$t-z = u$$
$$dt = du$$
Prepoctu i meze!
$$f(z) = \int_{-1}^1 \frac{1}{t-z} dt = \int_{-1-z}^{1-z} \frac{1}{u} du =$$
Opet z prvaku si pamatujeme, ze tohle je vzorec pro prirozeny logaritmus. Kdyztak si to pripiste na tahak nekam, tedy:
$$ = \left[ ln_0 u \right]_{-1-z}^{1-z} =$$
\textbf{Pozor!} K prirozenemu logarimu jsem dopsala $_0$, to oznacuje hlavni vetev komplexniho logaritmu. Musime pracovat s komplexnim logartimem, protoze promena $z$ vyskytujici se v substituci je komplexni, tedy i promena $u$ musi byt komplexni.

Dosadim meze:
$$ = ln_0(1-z) - ln_0 (-1-z) = $$
V dalsim kroku mohu rozepsat hlavni vetev logaritmu podle tahaku:
$$ = ln |1-z| + j \operatorname{arg}(1-z) - (ln|-1-z| + j \operatorname{arg}(-1-z)) = $$
Odstranim zavorku a take pozoruji, ze velikost $|1-z|$ a $|-1-z|$ jsou stejne, tedy logaritmy (zde uz realne) se odectou a zbyde jen:
$$ = j \operatorname{arg}(1-z) -j \operatorname{arg}(-1-z) = j (\operatorname{arg}(1-z)- \operatorname{arg}(-1-z))$$

Zde muzeme odpovedet na zadane otazky:

\subsection{a) Je $Re f(z)$ omezena na $\mathbb{C} \backslash [-1,1]$?}
Vidime, ze funkce nema zadnou realnou cast. Ale nevim, jak oduvodnit, ze je tedy neomezena (viz vysledky). Zde by to chtelo rozsirit. (TODO)

\subsection{b) Je $Im f(z)$ omezena na $\mathbb{C} \backslash [-1,1]$?}

Funkce $\operatorname{arg}$ znamena uhel komplexniho cisla. Jedna se tedy o rozdil dvou uhlu, z toho plyne, ze imaginarni cast je omezena.

\subsection{c) Je $f$ holomorfni na $\mathbb{C} \backslash [-1,1]$?}

TODO: rozsirit

