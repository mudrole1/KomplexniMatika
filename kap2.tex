\chapter{Zakladni pojmy analyzy v $\mathbb{C}$}

\section*{Vzorce}
\begin{itemize}
\item otevrena mnozina $G$ - s kazdym svym bodem $z \in G$ obsahuje i jiste jeho okoli (takze receno normalne, nejde rici, kde presne je hranice, protoze okoli je nedefinovane velke/male)
\item hranicni bod mnoziny M - pro kazde okoli $u(z,\varepsilon)$ plati:
$$u(z,\varepsilon) \bigcap M \neq 0 \wedge u(z,\varepsilon) \bigcap ( \mathbb{C} \backslash M) \neq 0$$
(tedy okoli hranicniho bodu $z$ nelezi ani v mnozine $M$, ani ve zbytku prostoru (tedy $\mathbb{C}\backslash M$), ucitel to vysvetloval, jako by ten bod byl rozkroceny, jednou nohou v M, jednou nohou venku... ale cely nepatri nikam
\item hranice = mnozina vsech hranicnich bodu
\item uzaver $\bar{M}$ mnoziny $M$ je definovan jako $M \bigcup \delta M$
\item pro uzavrene mnoziny plati, ze $M = \bar{M}$
\item priklady: $\emptyset, \mathbb{C}$ - otevrena mnozina
\item nesouvisla mnozina $M$ - existuji 2 disjunktni (nemaji prunik) otevrene mnoziny takove, ze pokryji mnozinu $M$ a neprekryvaji se
\item oblast - souvisla otevrena mnozina
\item konvexni mnozina - kazde 2 body lze spojit useckou, ktera lezi v mnozine (tedy kruh, ctverec, ale ne srdce!)
\item jednoznacne souvisla mnozina - stereograficka projekce mnoziny na riemanovu sferu ma souvisly doplnek (tedy zbytek na kouli je souvisly)
\end{itemize}

\newpage

\section{Overte, ze plati:}
\subsection{a) $|e^z| = e^{Re(z)}$}
Hratky s upravami, komplexni promenou $z$ muzu nahradit $z = x+jy$:
$$|e^{x+jy|}=$$
Scitani exponentu znamena nasobeni zakladu, tedy to muzu roztrhnout na dva cleny:
$$=|e^x \cdot e^{jy}|=$$
protoze je mezi cleny nasobeni, muzeme absolutni hodnotu roztrhnout na dve:
$$=|e^x|\cdot |e^{jy}| = $$
Ve clenu $e^{jy}$ vidime cast komplexniho cisla v exponencialnim tvaru. V tomto tvaru je velikost cisla dana pred exponentem, ale tady zadny dalsi clen neni, resp je tam jedna. Tedy velikost tohoto komplexniho cisla je 1, plati tedy ze:
$$=|e^x| \cdot 1 =$$
A velikost $e^x$ je to same (je to jen realna slozka, neni to komplexni cislo! Tedy:
$$= e^x$$
A my jsme tu pismenem $x$ oznacovali realnou slozku, tedy rovnost ze zadani plati.


\subsection{b) $\lim_{z \to \infty} e^z$ neexistuje}
Posleme nekonecno nejdriv po realne ose, a pak po imaginarni. Kdyz by limity byly stejne, dostali jsme se do stejneho "bodu", tedy limita je. Kdyz limity budou jine, takova limita neexistuje. Tedy po realne ose:
$$\lim_{x \to \infty} e^z = \lim_{x \to \infty} e^{x+jy} = lim_{x \to\infty} e^x \cdot e^{jy} = \infty$$
Kdyz $x$ posleme k nekonecnu, tak $e^x$ bude nekonecno krat cokoliv je zase nekoneno.

Po Imaginarni ose:
$$\lim_{y \to \infty} e^z = \lim_{y \to \infty} e^x \cdot e^{jy} =$$
Pozor! Tvar $e^{jy}$ je tvar komplexniho cisla v exponencialnim tvaru, muzeme to teda rozepsat jako:
$$= lim_{y \to \infty} e^x (\operatorname{cos}(y)+j\operatorname{sin}(y))$$
Pri tomto rozpisu vidime, ze hodnota $y$ se promita do vysledku periodicky. Kdyz si vezmete klasickou peridickou funkci cos, ci sin, tak pro nekonecnou hodnotu argumentu je hodnota v intervalu $<-1;1>$ Tedy i kdyz to prenasobime hodnotou $e^x$, ktera je omezena, takze limita celkova je taky omezena. 

Tedy vidime, ze limity se nerovnaji, tedy celkova limita neexistuje.

\subsection{c) $e^{\operatorname{Ln}(z)} = z$}
V exponentu je funkce $\operatorname{Ln}(z)$, vsimnte si, ze to je logaritmus s velkym L, takova funkce se rovna (pokud nemate jeste na tahaku, tak pripsat!)
$$\operatorname{Ln}(z) = ln|z|+j\operatorname{arg}z+2k\pi j$$, tedy fce ma realnou a imaginarni cast a navic je periodicka. Dosadime do zadani:
$$e^{\operatorname{Ln}(z)} = e^{ln|z|+j\operatorname{arg}z+2k\pi j} = e^{ln|z|}\cdot e^{j\operatorname{arg}z}\cdot e^{2k \pi j}$$

Vyresime postupne clen po clenu. Prvni clen $e^{ln|z|}$, exponencialni funkce je opacna k prirozenemu logaritmu. Tedy se to zaroven "vykrati" a zbyde jen $|z|$. Druhy clen: $\operatorname{arg}z$ znamena argument komplexniho cisla $z$, tedy muzeme oznacit i jako $\varphi$.  Tedy to muzeme zapsat jako vyraz $e^{j\varphi}$. Treti vyraz je $e^{jn}$, kdy $n = 2k \pi$, tedy opet exponencialni tvar komplexniho cisla. Kdyz to rozepiseme, tak je to:
$$e^{j2k\pi} = \operatorname{cos}(2k\pi)+j \operatorname{sin}(2k\pi)$$,
kdy $k$ je cele cislo, tedy cosinus se rovna 1, sinus 0 (pro uhel $2\pi$ a jeho cele nasobky). Tedy cely treti clen se rovna 1. Kdyz vsechny mezivysledky spojime do jednoho, ziskame:

$$e^{\operatorname{Ln}(z)} = |z|\cdot e^{j\varphi} \cdot 1 = |z|\cdot e^{j\varphi}$$
Coz je exponencialni tvar komplexniho cisla a jde to tedy prepsat na:
$$e^{\operatorname{Ln}(z)} = z$$
Tedy rovnost ze zadani plati.

\subsection{d) $e^{\frac{1}{2}\cdot \operatorname{Ln}(z)} = \sqrt{z}$}

Nasobeni exponentu je mocneni zakladu, tedy to muzeme prepsat jako:

$$e^{\frac{1}{2}\cdot \operatorname{Ln}(z)} = \left( e^{\operatorname{Ln}(z)} \right)^\frac{1}{2} = $$
Z prikladu vyse, vime, ze $e^{\operatorname{Ln}(z)} = z$, tedy to dosadime:
$$= (z)^\frac{1}{2} = $$
a exponentu $\frac{1}{2}$ se prepise jako odmocnina, tedy:
$$ \sqrt{z}$$
A tedy opet rovnost plati.

\subsection{e) $\operatorname{sin}^2(z) + \operatorname{cos}^2(z) = 1$}

V tahaku byste meli mit vzorecek na prepis sinu a kosinu, ale pozor s komplexnim argumentem, tedy se to da prepsat jako:
$$\left( \frac{e^{jz}-e^{-jz}}{2j} \right)^2 + \left( \frac{e^{jz}+e^{-jz}}{2}\right)^2 = \frac{e^{2jz}-2e^{jz}e^{-jz}+e^{-2jz}}{4j^2}+\frac{e^{2jz}+2e^{jz}e^{-jz}+e^{-2jz}}{4}=$$
V prnim jmenovateli je clen $j^2 = -1$, tedy zmenime znamenka v citateli prvniho zlomku. Oba jmenovatele budou uz jen $4$, tedy to muzeme napsat na spolecny jmenovatel:
$$= \frac{-e^{2jz}+2e^{jz}e^{-jz}-e^{-2jz}+e^{2jz}+2e^{jz}e^{-jz}+e^{-2jz}}{4} = \frac{4e^{jz}e^{-jz}}{4} = e^{jz-jz} = e^0 = 1$$
Opet jsme dokazali, ze rovnost plati.

\subsection{f) $\operatorname{sin}(2z) = 2\cdot \operatorname{sin}z \cdot \operatorname{cos}z$}
Tady zacneme s pravou stranou (je to jednodussi), pouzijeme stejne rozkladove vzorce jako v predchozim prikladu:
$$2\cdot \operatorname{sin}z \cdot \operatorname{cos}z = 2\cdot \frac{e^{jz}-e^{-jz}}{2j} \cdot \frac{e^{jz}+e^{-jz}}{2} = $$
Protoze citatele maji tvar $(a-b)(a+b)$ muzeme pouzit vzorec:
$$(a-b)(a+b) = a^2 - b^2$$
Tedy: (jeste se zkrati dvojky)
$$ = \frac{e^{2jz - e^{-2jz}}}{2j} $$

Pro levou stranu funkce $\operatorname{sin}(2z)$ lze rozepsat pomoci stejneho rozkladove vzorce, jen musime dat pozor na jiny argument $2z$! Tedy leva strana:
$$\operatorname{sin}(2z) = \frac{e^{j2z}-e^{-j2z}}{2j}$$

Vidime, ze prava i leva strana se rovnaji, dokazali jsme zadani.



