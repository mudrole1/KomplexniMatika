\chapter{Reprezentace holomorfni funkce Laurentovou radou}

\section*{Motivace}

Uz vime, (tusime :) ), jak holomorfni funkci aproximovat mocninou radou. Dulezity poznatek je vsak ten, ze mocnina rada konverguje na kruhu s polomerem konvergence R a stredem v $z_0$. Jenze puvodni holomorfni funkce je casto definovana i na jinych mnozinach (castech prostoru) nez je kruh, viz nasledujici priklad.

Uvazujte funkci
$$f(z) = \frac{1}{1-z}$$

Tato funkce je holomorfni v mnozine $\mathbb{C} \setminus \{1 \}$ (pro bod $z=1$ neni definovana). Pokud nyni vyjadrime jeji rozvoj (napr Taylorem) v mocninou radu (uvazujme napr rozvoj pro $z_0 = 0$, tak ziskame:
$$\frac{1}{1-z} = 1 + z + z^2 + z^3 + \dots = \sum_{n=0}^\infty z^n$$
coz je geometricka rada.

Pokud vyhodnotime polomer konvergence:
$$\frac{1}{R} = \lim_{n \to \infty} \sqrt[n]{a_n} = \lim_{n \to \infty} \sqrt[n]{1} = 1$$
Tedy rada konverguje pro takova $z$ lezici v kruhu o polomeru $R = 1$, tedy $|z|<1$. Pozor! Nepatri sem hranicni kruznice.

Vidime tedy, ze rozvojem funcke do mocnine rady jsme ziskali aproximaci jen v tomto kruhu. Ale funkce je definovana a holomorfni i mimo nej krome bodu $z = 1$. Jak reprezentovat funkci mimo kruh? Zde neni mozno uz pouzit jen mocninou radu a pouzivame tedy Laurentovy rady.

Tedy Laurentova rada je zobecnenim rady mocnine, tedy umoznuje aproximovat funkci radou na vetsich mnozinach (castech prostoru) nez je jenom kruh.
\section*{Vzorce}

\subsection*{Definice Laurentovy rady:} 
\begin{equation}
\label{eq:lau_def}
\sum_{-\infty}^\infty a_n (z-z_0)^n
\end{equation}
Na prvni pohled to muze vypadat jako mocnina, ale pozor! $n$ zacina od minus nekonecna! (u mocninych od nuly). Tedy cast vypada jako:
$$\dots + a_{-2} \frac{1}{(z-z_0)^2} + a_{-1}\frac{1}{z-z_0} + a_0(z-z_0)^0 + a_1(z-z_0)+ a_2 (z-z_0)^2 + \dots$$
Bod $z_0$ se nazyva stredem konvergence.
Laurentova rada ma hlavni cast (zaporne mocniny) a regularni cast (kladne mocniny, odpovida mocnine rade). Jeji konvergenci zapisujeme jako $P(z_0,r,R)$, kde $r$ je polomer konvergence hlavni casti, $R$ casti regularni. Z toho plyne, ze se muze stat, ze rada konverguje na mezikruzi (prostor mezi dvema kruhy).

Specialni pripad:
$$\sum_{-\infty}^\infty \frac{a_n}{z^n}$$
Je Laurentova rada se stredem v nekonecnu ($z_0 = \infty$).

\subsection*{Integralni vyjadreni koeficientu}
Koeficienty $a_n$ lze vypocitat jako:
\begin{equation}
\label{eq:lau_koef}
a_n = \frac{1}{2 \pi j} \int_{c} \frac{f(z)}{(z-z_0)^{n+1}}dz
\end{equation}
 kdy $c$ lezi v danem mezikruzi, kde rada konverguje.
Tento vzorec se hodi pro priklady tipu "Urcete koeficient u mocniny $z^{18}$.

\subsubsection*{Caychyuv vzorec pro mezikruzi}
$c_1, c_2$ jsou kladne orientovane Jordanovy krivky, kdy plati, ze $c_1$ je uvnitr $c_2$, zapis: $c_1 < Int c_2$:
\begin{equation}
\label{eq:lau_cauch}
f(z) = \frac{1}{2\pi j} \left( \int_{c_2} \frac{f(z)}{z-z_0}dz - \int_{c_1}\frac{f(z)}{z-z_0}dz \right)
\end{equation}

\subsection*{Singularity}
Singularita je vlastne bod $z_s$, ve kterem fce neni holomorfni (ale radeji se mrknete na poradnou teorii). Rozeznavame:
\begin{itemize}
\item odstranitelna - jestlize existuje vlastni limita f v bode $z_s$. Vsechny cleny v hlavni casti Laurentovy rady jsou nulove (tedy zbyde jen mocnina rada)
\item pol funkce - jestlize $lim_{z \to z_s} = \infty$, take urcujeme jeho rad $k$ (nasobnost). Laurentovy rada ma omezeny pocet prvku v hlavni casti. Cleny $a_{-k}, a_{-k+1}, atd$ ($n>=k$) jsou nenulove, vsechny koeficienty pro $n<k$ jsou nulove. 
\item podstatna - jestlize f nema limitu v bode $z_s$ - nekonecne mnoho koeficientu v hlavni casti Leurentovy rady
\end{itemize}

Specialni pripad - nekonecno ($\infty$) je pol funkce f - f je holomorfni na prstencovem okoli nekonecna. 

\subsection*{Reziduum}

Reziduum je jenom "husty" nazev pro koeficient $a_{-1}$. Tento koeficient ma pekne vlastnosti a usnadnuje vypocet mnoha veci, viz dalsi kapitola.

Obecne z integralniho vyjadreni vypoctu koeficientu ziskam:
\begin{equation}
\label{eq:rez_int}
a_{-1} = \frac{1}{2\pi j} \int_c f(z) dz
\end{equation}

Pokud $z_0$ je pol o nasobnosti k, reziduum spocteme jako:
\begin{equation}
\label{eq:rez_polk}
a_{-1} = res_{z_0} f(z) = \lim_{z \to z_0} \frac{1}{(k-1)!}\left[ (z-z_0)^k f(z)\right]^{(k-1)}
\end{equation}
kde $(k-1)$ znamena derivaci "k-1" krat.

Pro pol nasobnosti 1:
\begin{equation}
\label{eq:rez_pol1}
a_{-1} = res_{z_0} f(z) = \lim_{z \to z_0} (z-z_0)f(z)
\end{equation}

\subsubsection*{Vypocet rezidua pro nasobeni ci podil dvou funkci}
f(z), g(z) jsou holomorfni v $z_0 \in \mathbb{C}$, $z_0$ je jednonasobny koren g(z) (koren znamena, ze pro tento bod se fce rovna nule, tedy vlastne cely zlomek dvou funkci je v ohrozeni timto korenem), pak plati:
\begin{equation}
\label{eq:rez_pod}
rez_{z_0} \frac{f(z)}{g(z)} = \frac{f(z_0)}{g'(z_0)}
\end{equation}

f(z) holomorfni v $z_0$, g ma v $z_0$ jednonasobny pol, pak:
\begin{equation}
\label{eq:rez_nas}
rez_{z_0} f(z)g(z) = f(z_0) \cdot rez_{z_0} g(z)
\end{equation}

\subsubsection*{Specialni pripady}

Pro singularitu v nekonecnu, tedy $z_0 = \infty$
\begin{equation}
\label{eq:rez_nekon}
\int_c f(z) dz = 2\pi j res_\infty f(z)
\end{equation}

Funkce f ma v nekonecnu odstranitelnou singularitu:
\begin{equation}
\label{eq:rez_nek_ods}
rez_\infty f(z) = \lim_{z \to \infty} z^2 f'(z) = \lim z(f(\infty) - f(z))
\end{equation}

Funkce f ma v nekonecnu pol radu k:
\begin{equation}
\label{eq:rez_nek_polk}
rez_\infty f(z) = \frac{(-1)^k}{(k+1)!}\lim_{z \to \infty} (z^{k+2} f(z)^{(k+1)})
\end{equation}

Odstranitelna singularita ve vlastnim bode - reziduum je nulove.\\
Odstranitelna singularita v nevlastnim bode - reziduum libovolne.\\


\subsection*{Ostatni}
Uzitecne vzorecky:
$$\operatorname{cos}(t) = \frac{z^2+1}{2z}$$
$$\operatorname{sin}(t) = \frac{z^2-1}{2jz}$$
kde $z = e^{jt}$

TIP: Racionalni funkci (zlomek) rozlozime na parcialni zlomky v tvaru $\frac{A}{(z-a)^k}$, rozvineme fci $\frac{A}{z-a}$, a pak zderivujeme $k-1$ krat.

