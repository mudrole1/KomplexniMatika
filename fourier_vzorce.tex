\chapter{Fourierova transformace}

\section*{Potrebne vzorce}

\subsection*{Prima a inverzni Fourierova transformace}
\begin{equation}
\label{eq:definice}
\hat{f}(p) = \int_{-\infty}^\infty f(t) e^{-jpt} dt \; ; p \in R ; \; \operatorname{definice}
\end{equation}

\begin{equation}
\label{eq:inv_ttop}
\check{f}(p) = \frac{1}{2\pi}\int_{-\infty}^{\infty} f(t) e^{jpt} dt \; ; \operatorname{inverzni}
\end{equation}
(\textit{pozn.: Za definicni obor povazujeme mnozinu vsech p, pro ktere existuji prislusne integraly}).

Mezi primou a inverzni Fourierovou transformaci plati vztah:
\begin{equation}
\label{eq:pain}
\hat{f}(p) = 2\pi \check{f}(-p)
\end{equation}

Postacujici podminka pro existenci Fourierovy transformace: 
\begin{equation}
\int_{-\infty}^\infty |f(t)|dt < \infty
\end{equation}

\subsection*{Fourierovy obrazy racionalnich funkci - aplikace reziduove vety:}
\begin{equation}
\label{eq:rez}
\hat{f}(p)=\int_{-\infty}^{\infty} \frac{P(t)}{Q(t)}e^{-jpt}dt = \frac{2 \pi j}{|p|}\sum \operatorname{res}_z R(z)e^{jz}
\end{equation}
(\textit{pozn.: u Fourierovy transformace pouzijeme pouze ta rezidua, ktera maji kladnou imaginarni slozku!})

Pokud Q ma pouze realne koreny, pak $R(z)=\frac{P(z)}{Q(z)}$, nebo-li namisto $t$ dosadime z. Rezidua zname, jsou to koreny, ktere jsme urcili.
 
Pokud Q nema realne koreny, pak
\begin{equation}
\label{eq:rac_imag}
R(z)=\frac{P(-\frac{z}{p})}{	Q(-\frac{z}{p})}
\end{equation} 
a musime urcite rezidua.

\subsection*{Veta o inverzni Fourierove transformaci:}
\begin{equation}
\label{eq:invF}
f(t) = \frac{1}{2\pi}\int_{-\infty}^{\infty} \hat{f}(p)e^{jpt}dp
\end{equation}

\subsection*{Vzorce pro upravy funkci}

\begin{equation}
\label{eq:p_vzoru}
F\{f(t-a)\} = e^{-jpa} \hat{f}(p) \; ; \operatorname{posun\; ve \; vzoru}
\end{equation}

\begin{equation}
\label{eq:zm_meritka}
F\{f(at)\} = \frac{1}{|a|}\hat{f}\left(\frac{p}{a}\right), \; a \neq 0 \; ; \operatorname{zmena \; meritka}
\end{equation}

\begin{equation}
\label{eq:konjug}
F\{\overline{f(-t)}\} = \overline{\hat{f}(p)}\; ; \operatorname{pravidlo \; konjugace}
\end{equation}
Symbol $\overline{a}$ znamena komplexne sdruzene cislo, tedy pokud $a=x+jy$, $\overline{a}=x-jy$

\begin{equation}
\label{eq:p_obrazu}
F\{e^{jat}\cdot f(t) \} = \hat{f}(p-a) \; ; \operatorname{posun \; obrazu}
\end{equation}

\begin{equation}
\label{eq:der_obr}
F\{t\cdot f(t)\}(p) = j\frac{d}{dp}\hat{f}(p) \; ; \operatorname{derivace \; obrazu}
\end{equation}

Pokud f(t) je spojite diferenciovatelna fce a $f, f' \in L^1(R)$, pak:
\begin{equation}
\label{eq:obr_derivace}
\begin{array}{rcl}
F\{ f'(t)\} & = & jp\hat{f}(p)\; ; \operatorname{obraz \; derivace}\\
F\{ f^{(n)}(t) \} & = & (jp)^n \hat{f}(p)\\
\end{array}
\end{equation} 

Konvoluce:
\begin{equation}
\label{eq:konvoluce}
\begin{array}{rcl}
(f*g) & =& \int_{-\infty}^{\infty} f(s)g(t-s)ds \; ; \operatorname{definice \; konvoluce}\\
\hat{h}(p)&  = & \hat{f}(p)\cdot \hat{g}(p) \; ; \operatorname{obraz \; konvoluce} \\
\end{array}
\end{equation}


\subsection*{Uzitecne vzorce navic:}

Funkce $f(t) = e^{-t^2}$ ma Fourieruv obraz:
\begin{equation}
\label{eq:ena2}
\hat{f}(p)=\int_{-\infty}^\infty e^{{-t}^2}\cdot e^{-jpt}dt = \sqrt{\pi}e^{-\frac{p^2}{4}}
\end{equation}
Funkce $f(t) = e^{-at^2}$ ma Fourieruv obraz:

\begin{equation}
\label{eq:aena2}
\hat{f}(p)=\int_{-\infty}^\infty e^{-at^2}\cdot e^{-jpt}dt = \frac{\sqrt{\pi}}{\sqrt{a}}e^{-\frac{p^2}{4a}}
\end{equation}
Funkce $f(t) = e^{-a|t|}$
\begin{equation}
\label{eq:enaabs}
\hat{f}(p)=\frac{2a}{p^2+a^2}
\end{equation}